\documentclass[10pt, xcolor=table, x11names]{beamer}
\usepackage[spanish]{babel} %CORTE DE PALABRAS RESPETANDO EL IDIOMA ESPAÑOL.
\usepackage[Utf8]{inputenc} %acentos desde el teclado
\usepackage	{textpos}
\usepackage{tikz}
\usetikzlibrary{arrows,positioning} 
\usefonttheme{professionalfonts} % fuentes de LaTeX\epsilon
\usetheme{Boadilla}      % or try Darmstadt, Madrid, Warsaw, ...
\usecolortheme[RGB={130,130,190}]{structure} % or try albatross, beaver, crane, ...
\useinnertheme{rounded}
%\useoutertheme{shadow}
\setbeamertemplate{blocks}[rounded][shadow=true]
\setbeamertemplate{navigation symbols}{}
\setbeamercovered{transparent} % Velos
\setbeamertemplate{caption}[numbered]
\usepackage[spanish, authoryear, roud, datebegin]{flexbib} %CITAS BIBLIOGRÁFICAS
\newtheorem{Teorema}{Teorema}
\usepackage{ragged2e}
\justifying
\usepackage{booktabs}
\usepackage{multirow}
\usepackage[x11names,table]{xcolor}
\usepackage[pdftex]{graphicx}
\usepackage{epstopdf} % Convertir .eps a .pdf (si fuera necesario)
\DeclareGraphicsExtensions{.pdf,.png,.jpg, .eps} % busca en este orden!
\title[]{ENFOQUE DE ORGANIZACIÓN INDUSTRIAL (IO): COMPETENCIA PERFECTA}
\author[Luis Ortiz]{Luis Ortiz Cevallos}
\institute[SECMCA]{\bf SECMCA}
\date[\today]{\footnotesize \today}
\usepackage[pdftex]{hyperref}
\hypersetup{colorlinks,%
	citecolor=blue,%
	filecolor=blue,%
	linkcolor=blue,%
	urlcolor=blue,%
	pdftex}

\begin{document}


\begin{frame}
\titlepage
\end{frame}


\begin{frame}
    \frametitle{{\normalsize MODELO DE COMPETENCIA PERFECTA} {}}
    
    \begin{block} {Estructura básica del modelo}
        \begin{description}
            \item[Supuesto 1]  La actividad bancaria se modela como la producción de depósitos y créditos.
            \item[Supuesto 2] La tecnología bancaria está representada por una función de costos: $C(D,L)$ interpretado como la tecnología de manejar un volumen de depósitos D y créditos L.
            \item[Supuesto 3] Existen N bancos diferentes (indexados por $n=1, 2, \ldots n$) con la misma función de costos que satisface los habituales supuestos de convexidad:
            \begin{enumerate}
                \item Rendimiento decrecientes de escala 
                \item Regularidad (dos veces diferenciable)
            \end{enumerate}
            \item[Supuesto 4] La típica hoja de balance bancario es:
              \begin{table}[htbp]
            \centering
             \begin{tabular}{ll}
                ACTIVOS & PASIVOS \\
                Reservas $R_{n}$ & Depósitos $D_{n}$ \\
                Crédito  $L_{n}$ &  \\
            \end{tabular}%
            \end{table}%
            
                 
        \end{description}
        
    \end{block}	
    
\end{frame}


\begin{frame}\frametitle{{\normalsize MODELO DE COMPETENCIA PERFECTA} {}}
    
    Es de notar que la diferencia entre el volumen de depósitos y el crédito que los bancos han colocado (lo que hemos llamado reservas $R_{n}$), es divido en dos partes:
    \begin{itemize}
        \item Reservas de efectivo (disponibilidades) $C_{n}$
        \item Posición neta de los bancos sobre el mercado interbancario $M_{n}$
    \end{itemize}
    La diferencia entre $C_{n}$ y $M_{n}$ es que la primera no devenga intereses y por tanto la selección óptima implica que ésta sea en un mínimo nivel definido por el regulador.\\
    Definimos que:
    \begin{align}
    C_{n}&=\alpha D_{n}\nonumber
    \end{align} 
    Noten que $\alpha$ es un instrumento de política monetaria.
    
    


            
\end{frame}


\begin{frame}\frametitle{{\normalsize MODELO DE COMPETENCIA PERFECTA} {}}

\begin{block} {Estructura básica del modelo}
\begin{description}
    \item[Supuesto 5] Definimos tres agentes adicionales:
    \begin{enumerate}
        \item El gobierno que incluye al banco central
        \item Las firmas
        \item Hogares
    \end{enumerate}  
    \item Los bancos sirven de intermediarios recolectando el ahorro de los hogares S con el cual financian la inversión I de las firmas.
    \item El Gobierno financia su gasto G, emitiendo títulos de deuda B y dinero $M_{b}$.
    \item $M_{b}$ es utilizado por los bancos para financiar sus $C_{n}$.
     
     % Table generated by Excel2LaTeX from sheet 'Hoja1'
        
\end{description}

\end{block}	

\end{frame}

\begin{frame}\frametitle{{\normalsize MODELO DE COMPETENCIA PERFECTA} {}}

vinculación entre sectores.
\begin{center}
	\begin{table*}
		\caption{}
		\begin{tabular}{|cc|cc|}
			\toprule
			\rowcolor[rgb]{ 0,  .439,  .753} \multicolumn{2}{|c|}{Gobierno} & \multicolumn{2}{c|}{Hogares} \\
			\midrule
			\rowcolor[rgb]{ .608,  .761,  .902} ACTIVOS & PAS/CAP & ACTIVOS &  PAS/CAP  \\
			G    & B    & B & S \\
			$Rin_{b}$     & $M_{b}$     & D   &  \\
			$L_{b}$ & Señoreaje  &     &  \\
			  & -Transferencia  &     &  \\
			\midrule
			\rowcolor[rgb]{ 0,  .439,  .753} \multicolumn{2}{|c|}{FIRMAS} & \multicolumn{2}{c|}{Bancos} \\
			\midrule
			\rowcolor[rgb]{ .608,  .761,  .902} ACTIVOS &  PAS/CAP  & ACTIVOS &  PAS/CAP \\
			I     & L    &  $M_{b}$     & D \\
			      &       & L           &   \\
			
			\bottomrule
		\end{tabular}%
	\end{table*}%
\end{center}

\end{frame}


\begin{frame}\frametitle{{\normalsize MODELO DE COMPETENCIA PERFECTA} {}}


Con estos supuesto tenemos las siguientes identidades:
\begin{align}
Dinero&=D= \sum_{n=1}^{N}D_{n} \nonumber \\ 
M_{b}&=\sum_{n=1}^{N}C_{n}=\alpha D   \nonumber
\end{align} 

\end{frame}


\begin{frame}\frametitle{{\normalsize MODELO DE COMPETENCIA PERFECTA: EL ENFOQUE DEL MULTIPLICADOR DEL CRÉDITO} {}}
 Dado que (enfoque macro):
  \begin{align}
  M_{b}&=\alpha D   \\ \nonumber
  D&=\frac{M_{b}}{\alpha}=\frac{G-B}{\alpha}  \\ \nonumber
  L&=D-M_{b}=M_{b}\left(\frac{1}{\alpha}-1 \right)=\left(G-B \right)\left( \frac{1}{\alpha}-1\right)  
  \end{align}   
  El multiplicador monetario es definido como el efecto marginal de un cambio en la base monetaria sobre la cantidad de dinero.  
  \begin{align}
  \dfrac{\delta D}{\delta M_{b}}&=-\dfrac{\delta D}{\delta B}=\frac{1}{\alpha}>0 \nonumber
  \end{align}    
  Similarmente el multiplicador del crédito es definido como el efecto sobre el crédito de un cambio margina en la base monetaria.
  \begin{align}
  \dfrac{\delta L}{\delta M_{b}}&=-\dfrac{\delta L}{\delta B}=\frac{1}{\alpha}-1>0 \nonumber
  \end{align}   
  
\end{frame}




\begin{frame}\frametitle{{\normalsize MODELO DE COMPETENCIA PERFECTA: EL ENFOQUE DEL MULTIPLICADOR DEL CRÉDITO} {}}
   
   \begin{block} {Crítica}
    El problema con el enfoque del multiplicador de créditos es que los bancos son considerados como agentes pasivos. La política monetaria moderna se precisa como una intervención sobre la tasa de interés r, en el cual el banco central refinancia a los bancos comerciales. Esa intervención afecta el comportamiento de los bancos comerciales, quienes afecta las tasas de depósitos ($r_{d}$ ) y créditos ($r_{l}$). Para entender eso necesitamos un modelos del comportamiento individual de los bancos.      
       
   \end{block}	 
    
   
    
\end{frame}



\begin{frame}
    \frametitle{{\normalsize MODELO DE COMPETENCIA PERFECTA: Comportamiento de los bancos} {}}
    
    Los bancos son tomadores de precios, ellos toman como dado $r_{d}$, $r_{l}$ y la tasa interbancaria r. Por tanto el beneficio de los bancos está descrito por:
    
   \begin{align}
   \pi&=r_{L}L+rM-r_{D}D-C(D,L) \nonumber
   \end{align}    
    Donde M es la posición neta de cualquier banco en el mercado interbancario. El cual está dado por:
    \begin{align}
    M&=(1-\alpha)D-L \nonumber\\
    \pi(D,L)&=r_{L}L+r((1-\alpha)D-L)-r_{D}D-C(D,L) \nonumber \\
    \pi(D,L)&=(r_{L}-r)L+(r(1-\alpha)-r_{D})D-C(D,L) 
    \end{align} 
    Así el comportamientos de los bancos se deducen de las condiciones de orden:
    \begin{align}
    \dfrac{\delta \pi}{\delta L}&=(r_{L}-r)-\dfrac{\delta C}{\delta L}=0  \nonumber \\
    & \\
    \dfrac{\delta \pi}{\delta D}&=(r(1-\alpha)-r_{D})-\dfrac{\delta C}{\delta D}=0  \nonumber
    \end{align} 
    
    
    
\end{frame}




\begin{frame}
    \frametitle{{\normalsize MODELO DE COMPETENCIA PERFECTA: Comportamiento de los bancos} {}}
    \begin{block} {Resultado}
        \begin{enumerate}
            \item Un banco competitivo debe ajustar su volumen de crédito y depósitos de manera de que el margen de intermediación se iguale al manejo de sus costos.
            \item Un incremento de $r_{D}$ hace que los bancos disminuya su demanda de depósitos. Un incremento de $r_{L}$ hace que los bancos aumenten su oferta de crédito. El efecto cruzado depende del signo de:
          \begin{align}
          \dfrac{\delta^{2}C}{\delta D \delta L} \nonumber
          \end{align} 
          Si $\dfrac{\delta^{2}C}{\delta D \delta L}>0 $, un incremento en $r_{L}$ implica un decremento de D. Mientras un incremento de  $r_{D}$ implica un incremento de L (lo opuesto ocurre si $\dfrac{\delta^{2}C}{\delta D \delta L}<0 $ ).\\
           Cuando los costos sean separables $\dfrac{\delta^{2}C}{\delta D \delta L}=0 $ los efectos cruzados son nulos.   
        \end{enumerate}     
        
    \end{block}	 


\end{frame}



\begin{frame}
    \frametitle{{\normalsize MODELO DE COMPETENCIA PERFECTA: Comportamiento de los bancos} {}}
    La interpretación económica de la condición de $\dfrac{\delta^{2}C}{\delta D \delta L}$ nos conduce a la noción de economías de gama. \\
    Y es que cuando $\dfrac{\delta^{2}C}{\delta D \delta L}<0$ un incremento de L trae como consecuencia decrecer el costo marginal de los depósitos, lo que es una forma particular de economía de gama, ya que implica que el banco universal (quien ofrece tanto créditos como depósitos), es más eficientes que dos entidades separadas cada una especializada en un servicio (lo opuesto ocurre si $\dfrac{\delta^{2}C}{\delta D \delta L}>0 $ ).
\end{frame}

\begin{frame}
    \frametitle{{\normalsize MODELO DE COMPETENCIA PERFECTA: EQUILIBRIO} {}}
    Dado que son N bancos, cada uno de ellos está caracterizado por una oferta de créditos $L^{n}(r_{L}, r_{D}, r)$ y demanda de depósitos $D^{n}(r_{L}, r_{D}, r)$. A la vez  definimos $I(r_{L})$ como la demanda de inversión que realizan las firmas la cual es igual a la demanda de créditos dado que las firmas no pueden emitir deuda y $S(r_{D}) $ como la función de ahorro de los hogares (asumiendo que tanto los depósitos como los B, son perfectos sustitutos y por tanto, su tasa de interés es la misma). Entonces el equilibrio competitivo se caracteriza por las siguientes ecuaciones:
     \begin{align}
     I(r_{L})&=\sum_{i=1}^{N}L^{n}(r_{L}, r_{D}, r) \;\;\; \mbox{(mercado de crédito)}\\
     S(r_{D})&=B+\sum_{i=1}^{N}D^{n}(r_{L}, r_{D}, r) \;\;\; \mbox{(mercado de ahorro)}\\
     \sum_{i=1}^{N}L^{n}(r_{L}, r_{D}, r)&=(1-\alpha)\sum_{i=1}^{N}D^{n}(r_{L}, r_{D}, r) \;\;\; \mbox{(mercado interbancario)}
     \end{align} 
    
    
    
\end{frame}



\begin{frame}
    \frametitle{{\normalsize MODELO DE COMPETENCIA PERFECTA: EQUILIBRIO} {}}
   La ecuación 7 recoge el hecho de que la posición agregada de todos los bancos en el mercado interbancario es cero. De manera general un término que denote la inyección de dinero por el banco central puede ser adicionado a esa ecuación, en ese caso r sería la variable de política escogida del banco central. Alternativamente r podría ser determinada por el mercado de capital internacional, en ese caso se adiciona a 7 un termino de flujo neto de país. En ambos caso r sería exógeno y 7 desaparece.
   
   En el caso de costo de intermediación  marginales constantes ($C_{L}^{'}=\gamma_{L},\; C_{D}^{'}=\gamma_{D} $),  se obtiene una simple caracterización de equilibrio, en la que se sustituye 5 y 6 por una determinación directa de $r_{L}$ y $r_{D}$, deducidas de 4.
   \begin{align}
   r_{L}&=r+\gamma_{L} \\
   r_{D}&=r(1-\alpha)-\gamma_{D} 
   \end{align} 
    
\end{frame}



\begin{frame}
    \frametitle{{\normalsize MODELO DE COMPETENCIA PERFECTA: EQUILIBRIO} {}}
    Entonces la tasa de interés r del mercado interbancario es deducido de 7, la cual puede escribirse como:
    \begin{align}
    \sum_{i=1}^{N}L^{n}(r_{L}, r_{D}, r)&=(1-\alpha)\sum_{i=1}^{N}D^{n}(r_{L}, r_{D}, r)  \nonumber \\
    I(r_{L})&=(1-\alpha)(S(r_{D})-B) \nonumber \\
    S(r(1-\alpha)-\gamma_{D})-\frac{I(r+\gamma_{L})}{(1-\alpha)} &=B
    \end{align} 
    La ecuación 10, permite determinar los efectos macro de un cambio marginal en el coeficiente de reserva $\alpha$, o en B, sobre el nivel de equilibrio de $r_{L}$ y $r_{D}$ siendo estos resultados más complejos dado que se tiene en cuenta el comportamiento de los bancos.
\end{frame}

\begin{frame}\frametitle{{\normalsize MODELO DE COMPETENCIA PERFECTA: EQUILIBRIO} {}}
    
    \begin{block} {Resultado}
        \begin{enumerate}
            \item Una emisión de titulos de parte del gobierno, conlleva una caída en el crédito y depósitos, sin embargo la magnitud de su caída es menor con respecto al modelo estándar:
            \begin{align}
            |\frac{\delta D}{\delta B}| =1,\;\;\;\; |\frac{\delta L}{\delta B}|=1-\alpha \nonumber 
            \end{align} 
            \item Si $\alpha$ se incrementa el volumen de crédito decrece, pero sus efectos sobre los depósitos son ambiguos.
        \end{enumerate}    
   
    \end{block}	 
    
     La segunda parte de estos resultados puede parecer sorprendente, dado que la condición de primer orden establece que la tasa de interés de los depósitos es función decreciente de  $\alpha$. Pero como el mercado interbancario es endógeno, si en el extremo opuesto donde el mercado interbancario es exógeno, el r es controlado por el banco central, pudiendo ser que la $r_{L}$ no se vea afectada por $\alpha$, y sólo la tasa de depósito se ajuste.  
    
\end{frame}


\end{document}