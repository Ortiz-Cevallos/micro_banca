%=========================================================================
%
%  DETERMINE WHAT TO COMPILE
%
%-------------------------------------------------------------------------


\ifdefined\presentationMode
\ifcase\presentationMode\relax
\documentclass[x11names, handout]{beamer}
\or
\documentclass[x11names, trans]{article}
\usepackage{beamerarticle}
\else
\documentclass[x11names]{beamer}
\fi
\else
\documentclass[x11names]{beamer}
\fi


%=========================================================================
%
%  LOAD PACKAGES
%
%-------------------------------------------------------------------------
%\usepackage[x11names,table]{xcolor}
\definecolor{luis}{RGB}{175,160,225}
\usetheme{Boadilla}      % or try Darmstadt, Madrid, Warsaw, ...
\setbeamercolor{frametitle}{fg=white,bg=luis}


\usepackage{polyglossia}
\setmainlanguage{spanish}

\usepackage{todonotes}
\usepackage{fontspec,xunicode}

\usepackage{tabu, colortbl}
\usepackage{hyperref}
\hypersetup{
	bookmarks=true,         % show bookmarks bar?
	colorlinks=true,       % false: boxed links; true: colored links
	linkcolor=black,          % color of internal links
	citecolor=blue,        % color of links to bibliography
	filecolor=magenta,      % color of file links
	urlcolor=black           % color of external links
}


\usepackage{tikz}
\usetikzlibrary{arrows,shapes,positioning,shadows,trees}
\usepackage{pgfplots}
\pgfplotsset{width=10cm, compat=1.10}
\usepackage{ctable}
\newcolumntype{E}{>{$}l<{$}}
\usepackage{multirow}
%\usepackage{wasysym}
\usepackage{graphicx}
\usepackage{multimedia}

	
\usepackage[%
bibstyle=authoryear,%
citestyle=authoryear-icomp,%
sorting=nyt,%
backend=biber,%
url=false,%
uniquename=false,%
doi=false,%
bibencoding=utf8%
]{biblatex} % to format the bibliography
% bib resources are added in course-info.tex file.

%\usepackage{changepage}
\usepackage{etoolbox}
\newtoggle{HIDDEN}
\togglefalse{HIDDEN}

%#================ NEW MATH OPERATORS======================================
\usepackage{amsmath,amsfonts, amssymb} % tools for typing mathematics
\usepackage{xfrac}
%\usepackage{pstricks}
\DeclareMathOperator{\E}{\mathbb{E}}
\DeclareMathOperator{\dd}{\,d\!}
\DeclareMathOperator{\Var}{Var}
\DeclareMathOperator{\Cov}{Cov}
\DeclareMathOperator{\Corr}{Corr}
\DeclareMathOperator{\Prob}{\mathbb{P}}
\newcommand{\MAT}[1]{\begin{bmatrix} #1 \end{bmatrix}}
\newcommand{\argmin}[1]{\underset{#1}{\operatornamewithlimits{argmin}}}
\newcommand{\argmax}[1]{\underset{#1}{\operatornamewithlimits{argmax}}}
\newcommand{\marginal}[2]{\frac{\partial #1}{\partial #2}}
\newcommand{\notation}[2]{\underset{\scriptsize\color{SteelBlue4!60}\text{#2}}{#1}}
\newcommand{\given}[1]{\,|\,#1}

\newcounter{theexample} \setcounter{theexample}{0}

\AtBeginSection{\frame{\sectionpage}}
\AtBeginSubsection{\frame{\subsectionpage}}
\renewcommand{\sectionname}{Sección}
\renewcommand{\subsectionname}{Subsección}

%=========================================================================
%
%  COMMANDS FOR PRESENTATION MODE
%
%-------------------------------------------------------------------------
\mode<presentation>{
	
	\usefonttheme{professionalfonts}
	\renewcommand*{\thefootnote}{\fnsymbol{footnote}}
	
	\newcommand{\ITEM}[2]{%
		\begin{tabu} to \textwidth {>{\color{white}\columncolor{DodgerBlue4}}X[1,c,m]>{\small}X[4,l,m]}
			#1 & #2 \\
		\end{tabu}
	}
	
	\newcommand{\EQ}[2]{%
		\begin{tabu} to \textwidth {>{\color{white}\columncolor{teal}\scriptsize}X[1,c,m]>{\columncolor{orange!10}\small}X[4,c,m]}
			\textbf{#1} & \begin{equation*}   #2 \end{equation*}\\
		\end{tabu}
	}

	\newcommand{\CONCEPT}[4]{%
        \extrarowsep =0.75em
		\begin{tabu} to #1\textwidth {>{\color{white}\columncolor{teal}\scriptsize}X[1,c,m]>{\columncolor{teal!15}\small}X[#2,p,m]}
			\textbf{#3} &  #4 \\
		\end{tabu}
	}	

	\newenvironment{EXAMPLE}[1]{%
		\stepcounter{theexample}
		\setbeamercolor{background canvas}{bg=purple!25}
		\begin{frame}[plain,noframenumbering]
			\LARGE Ejemplo \arabic{theexample}:
			
			\huge #1
		\end{frame}
		\setbeamercolor{background canvas}{bg=purple!12}
		}{
		}

	\newenvironment{SIDENOTE}[1]{%
		\setbeamercolor{background canvas}{bg=orange!8}
		\begin{frame}{Side note: #1}
	}{
		\end{frame}
	}

}



%=========================================================================
%
%  COMMANDS FOR ARTICLE MODE
%
%-------------------------------------------------------------------------
\mode<article>{
	\usepackage[top=1in, bottom=1.25in, left=1.25in, right=1.25in]{geometry}
	
	\newcommand{\ITEM}[2]{%
		\begin{tabu} to \textwidth {X[1,c,m]X[4,l,m]}
			\hline
			#1 & #2 \\ \hline
		\end{tabu}
	}
	
	\newcommand{\EQ}[2]{%
		\begin{tabu} to \columnwidth {X[1,cm]X[4,lm]}
			\rowcolor{orange!10}
			#1 & \begin{equation*}   #2 \end{equation*}\\ 
		\end{tabu}
	}

	\usepackage{framed}
	\newenvironment{EXAMPLE}[1]{%
		\stepcounter{theexample}
		\begin{framed}
			\colorbox{teal}{\textcolor{white}{\Large Ejemplo
            \arabic{theexample}: #1}} 
			\vspace{1em}		\\
	}{
		\end{framed}
	}
	
	\newenvironment{SIDENOTE}[1]{%
	\begin{framed}
		\colorbox{orange!80!black}{\textcolor{white}{Side note: #1}} 
		}{
	\end{framed}
}

}



%=========================================================================
%
%  COMMANDS FOR ALL MODES
%
%-------------------------------------------------------------------------
    \makeatletter
    \patchcmd{\beamer@sectionintoc}{\vskip1.5em}{\vskip0.5em}{}{}
    \makeatother

\newcommand{\makeFrameTitle}{%
	\begin{frame}[plain,noframenumbering]%
	\begin{textblock*}
		\includegraphics[height=2.0cm]{imag/LUIS.png}\hfill\includegraphics[height=2.0cm]{imag/atlantida.png}
	\end{textblock*}
	\vspace{0.5cm}\maketitle%
	\end{frame}%
	
		
	\begin{frame}[plain,noframenumbering]{Contenido}    
		\setbeamertemplate{section in toc}[sections numbered]
		\setbeamertemplate{subsection in toc}[subsections numbered]	
		\tableofcontents
	\end{frame}
	
}

\newcommand{\contn}{\hfill(cont'n)}



\newcommand{\makeReferencesFrame}[1]{
	\begin{frame}[fragile,allowframebreaks]
		\nocite{#1}
		\printbibliography
	\end{frame}
}


\newcommand{\CASE}[1]{ \colorbox{Coral4}{\color{white} #1} \vspace{1em}}


\newcommand{\IMPORTANT}[1]{%
	\begin{tabu}  {>{\columncolor{DodgerBlue4!10}}X[c,m]}
		\begin{equation*} #1 \end{equation*}\\
	\end{tabu}
}


\newcommand{\NOTICE}[1]{%
	\begin{tabu} {>{\columncolor{Yellow1!30}}X[c,m]}
		\begin{equation*} #1 \end{equation*}\\
	\end{tabu}
}



\newcommand{\colorblock}[3]
{\begingroup
	\setbeamercolor{block title}{bg=#1,fg=white}
	\setbeamercolor{block body}{bg=#1!6}
	\begin{block}{#2}
		#3
	\end{block}
	\endgroup
}


%-------------------------------------------------------------------------

\newcommand{\rojitas}[1]{\textbf{{\color{red} #1}}} 
\newcommand{\negritas}[1]{\textbf{#1}} 
\newcommand{\espacio}{\vspace{1em}}
\newcommand{\itemdos}{\item[\checkmark]}
\newcommand{\flechas}{\item[$\rightarrow$]}


\newcommand{\ALTERNATE}[2]{%
	\only<article>{#1}\only<presentation>{#2}
}




