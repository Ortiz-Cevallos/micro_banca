\def\presentationMode{0}
%=========================================================================
%
%  DETERMINE WHAT TO COMPILE
%
%-------------------------------------------------------------------------


\ifdefined\presentationMode
\ifcase\presentationMode\relax
\documentclass[x11names, handout]{beamer}
\or
\documentclass[x11names, trans]{article}
\usepackage{beamerarticle}
\else
\documentclass[x11names]{beamer}
\fi
\else
\documentclass[x11names]{beamer}
\fi


%=========================================================================
%
%  LOAD PACKAGES
%
%-------------------------------------------------------------------------
%\usepackage[x11names,table]{xcolor}
\definecolor{luis}{RGB}{175,160,225}
\usetheme{Boadilla}      % or try Darmstadt, Madrid, Warsaw, ...
\setbeamercolor{frametitle}{fg=white,bg=luis}


\usepackage{polyglossia}
\setmainlanguage{spanish}

\usepackage{todonotes}
\usepackage{fontspec,xunicode}

\usepackage{tabu, colortbl}
\usepackage{hyperref}
\hypersetup{
	bookmarks=true,         % show bookmarks bar?
	colorlinks=true,       % false: boxed links; true: colored links
	linkcolor=black,          % color of internal links
	citecolor=blue,        % color of links to bibliography
	filecolor=magenta,      % color of file links
	urlcolor=black           % color of external links
}


\usepackage{tikz}
\usetikzlibrary{arrows,shapes,positioning,shadows,trees}
\usepackage{pgfplots}
\pgfplotsset{width=10cm, compat=1.10}
\usepackage{ctable}
\newcolumntype{E}{>{$}l<{$}}
\usepackage{multirow}
%\usepackage{wasysym}
\usepackage{graphicx}
\usepackage{multimedia}

	
\usepackage[%
bibstyle=authoryear,%
citestyle=authoryear-icomp,%
sorting=nyt,%
backend=biber,%
url=false,%
uniquename=false,%
doi=false,%
bibencoding=utf8%
]{biblatex} % to format the bibliography
% bib resources are added in course-info.tex file.

%\usepackage{changepage}
\usepackage{etoolbox}
\newtoggle{HIDDEN}
\togglefalse{HIDDEN}

%#================ NEW MATH OPERATORS======================================
\usepackage{amsmath,amsfonts, amssymb} % tools for typing mathematics
\usepackage{xfrac}
%\usepackage{pstricks}
\DeclareMathOperator{\E}{\mathbb{E}}
\DeclareMathOperator{\dd}{\,d\!}
\DeclareMathOperator{\Var}{Var}
\DeclareMathOperator{\Cov}{Cov}
\DeclareMathOperator{\Corr}{Corr}
\DeclareMathOperator{\Prob}{\mathbb{P}}
\newcommand{\MAT}[1]{\begin{bmatrix} #1 \end{bmatrix}}
\newcommand{\argmin}[1]{\underset{#1}{\operatornamewithlimits{argmin}}}
\newcommand{\argmax}[1]{\underset{#1}{\operatornamewithlimits{argmax}}}
\newcommand{\marginal}[2]{\frac{\partial #1}{\partial #2}}
\newcommand{\notation}[2]{\underset{\scriptsize\color{SteelBlue4!60}\text{#2}}{#1}}
\newcommand{\given}[1]{\,|\,#1}

\newcounter{theexample} \setcounter{theexample}{0}

\AtBeginSection{\frame{\sectionpage}}
\AtBeginSubsection{\frame{\subsectionpage}}
\renewcommand{\sectionname}{Sección}
\renewcommand{\subsectionname}{Subsección}

%=========================================================================
%
%  COMMANDS FOR PRESENTATION MODE
%
%-------------------------------------------------------------------------
\mode<presentation>{
	
	\usefonttheme{professionalfonts}
	\renewcommand*{\thefootnote}{\fnsymbol{footnote}}
	
	\newcommand{\ITEM}[2]{%
		\begin{tabu} to \textwidth {>{\color{white}\columncolor{DodgerBlue4}}X[1,c,m]>{\small}X[4,l,m]}
			#1 & #2 \\
		\end{tabu}
	}
	
	\newcommand{\EQ}[2]{%
		\begin{tabu} to \textwidth {>{\color{white}\columncolor{teal}\scriptsize}X[1,c,m]>{\columncolor{orange!10}\small}X[4,c,m]}
			\textbf{#1} & \begin{equation*}   #2 \end{equation*}\\
		\end{tabu}
	}

	\newcommand{\CONCEPT}[4]{%
        \extrarowsep =0.75em
		\begin{tabu} to #1\textwidth {>{\color{white}\columncolor{teal}\scriptsize}X[1,c,m]>{\columncolor{teal!15}\small}X[#2,p,m]}
			\textbf{#3} &  #4 \\
		\end{tabu}
	}	

	\newenvironment{EXAMPLE}[1]{%
		\stepcounter{theexample}
		\setbeamercolor{background canvas}{bg=purple!25}
		\begin{frame}[plain,noframenumbering]
			\LARGE Ejemplo \arabic{theexample}:
			
			\huge #1
		\end{frame}
		\setbeamercolor{background canvas}{bg=purple!12}
		}{
		}

	\newenvironment{SIDENOTE}[1]{%
		\setbeamercolor{background canvas}{bg=orange!8}
		\begin{frame}{Side note: #1}
	}{
		\end{frame}
	}

}



%=========================================================================
%
%  COMMANDS FOR ARTICLE MODE
%
%-------------------------------------------------------------------------
\mode<article>{
	\usepackage[top=1in, bottom=1.25in, left=1.25in, right=1.25in]{geometry}
	
	\newcommand{\ITEM}[2]{%
		\begin{tabu} to \textwidth {X[1,c,m]X[4,l,m]}
			\hline
			#1 & #2 \\ \hline
		\end{tabu}
	}
	
	\newcommand{\EQ}[2]{%
		\begin{tabu} to \columnwidth {X[1,cm]X[4,lm]}
			\rowcolor{orange!10}
			#1 & \begin{equation*}   #2 \end{equation*}\\ 
		\end{tabu}
	}

	\usepackage{framed}
	\newenvironment{EXAMPLE}[1]{%
		\stepcounter{theexample}
		\begin{framed}
			\colorbox{teal}{\textcolor{white}{\Large Ejemplo
            \arabic{theexample}: #1}} 
			\vspace{1em}		\\
	}{
		\end{framed}
	}
	
	\newenvironment{SIDENOTE}[1]{%
	\begin{framed}
		\colorbox{orange!80!black}{\textcolor{white}{Side note: #1}} 
		}{
	\end{framed}
}

}



%=========================================================================
%
%  COMMANDS FOR ALL MODES
%
%-------------------------------------------------------------------------
    \makeatletter
    \patchcmd{\beamer@sectionintoc}{\vskip1.5em}{\vskip0.5em}{}{}
    \makeatother

\newcommand{\makeFrameTitle}{%
	\begin{frame}[plain,noframenumbering]%
	\begin{textblock*}
		\includegraphics[height=2.0cm]{imag/LUIS.png}\hfill\includegraphics[height=2.0cm]{imag/atlantida.png}
	\end{textblock*}
	\vspace{0.5cm}\maketitle%
	\end{frame}%
	
		
	\begin{frame}[plain,noframenumbering]{Contenido}    
		\setbeamertemplate{section in toc}[sections numbered]
		\setbeamertemplate{subsection in toc}[subsections numbered]	
		\tableofcontents
	\end{frame}
	
}

\newcommand{\contn}{\hfill(cont'n)}



\newcommand{\makeReferencesFrame}[1]{
	\begin{frame}[fragile,allowframebreaks]
		\nocite{#1}
		\printbibliography
	\end{frame}
}


\newcommand{\CASE}[1]{ \colorbox{Coral4}{\color{white} #1} \vspace{1em}}


\newcommand{\IMPORTANT}[1]{%
	\begin{tabu}  {>{\columncolor{DodgerBlue4!10}}X[c,m]}
		\begin{equation*} #1 \end{equation*}\\
	\end{tabu}
}


\newcommand{\NOTICE}[1]{%
	\begin{tabu} {>{\columncolor{Yellow1!30}}X[c,m]}
		\begin{equation*} #1 \end{equation*}\\
	\end{tabu}
}



\newcommand{\colorblock}[3]
{\begingroup
	\setbeamercolor{block title}{bg=#1,fg=white}
	\setbeamercolor{block body}{bg=#1!6}
	\begin{block}{#2}
		#3
	\end{block}
	\endgroup
}


%-------------------------------------------------------------------------

\newcommand{\rojitas}[1]{\textbf{{\color{red} #1}}} 
\newcommand{\negritas}[1]{\textbf{#1}} 
\newcommand{\espacio}{\vspace{1em}}
\newcommand{\itemdos}{\item[\checkmark]}
\newcommand{\flechas}{\item[$\rightarrow$]}


\newcommand{\ALTERNATE}[2]{%
	\only<article>{#1}\only<presentation>{#2}
}





\usepackage{pgfpages}
\pgfpagesuselayout{2 on 1}[letterpaper,border shrink=5mm]


\author[Luis Ortiz Cevallos e-mail: \href{leortiz@uc.cl}{\textit{leortiz@uc.cl}}]{Profesor: Luis Ortiz Cevallos, e-mail:\href{leortiz@uc.cl}{\textit{leortiz@uc.cl}} }
\title[MICROECONOMÍA DE LA BANCA]{\vspace*{1.0em} MICROECONOMÍA DE LA BANCA}
\date[\href{https://ortiz-cevallos.github.io/luisortiz.github.io/ }{\textit{https://ortiz-cevallos.github.io/luisortiz.github.io/}}]{}



\addbibresource{econometria-referencias.bib}






%\includeonly{02-Conceptos-basicos}






\begin{document}
\makeFrameTitle

\section{Aversión y premio por riesgo}
\begin{frame}
\frametitle{Aversión al riesgo y premio de riesgo}

Un individuo se dice que es averso al riesgo si y sólo si:
\\

\textit{Su función de utilidad es concova, implicando que este sujeto no aceptaría participar en un ``juego justo'' (fair lottery).\\
     Un ``juego justo'' es definido como aquel cuyo valor esperado es cero.}\\

Un ejemplo:\\
Supongamos un ``juego justo'' que tiene un pago aleatoria de $ \hat{\epsilon}$, donde:

\[
\hat{\epsilon}= \left\{ \begin{array}{lcl}
h_{1} & con\; probabilidad & p \\
& & \\
h_{2} & con\; probabilidad & 1-p
\end{array}
\right.
\] 

Por tanto para ser un ``juego justo'' debe cumplirse:
\begin{align}
E(\hat{\epsilon})&=ph_{1}+(1-p)h_{2}=0 \nonumber \\
ph_{1}&=-(1-p)h_{2} \nonumber \\
\frac{h_{1}}{h_{2}}&=-\frac{(1-p)}{p}\nonumber
\end{align}    
\end{frame}


\begin{frame}
    \frametitle{Aversión al riesgo y premio de riesgo}
  Es de notar que si el individuo acepta el ``juego justo'', considerando una función de utilidad esperada a VNM tendría:
\begin{align}
V&=E(U(W+\hat{\epsilon}))\nonumber 
\end{align}    
 Mientras si no acepta participar en el juego justo tendría:
 \begin{align}
 V&=E(U(W))=U(W)\nonumber 
 \end{align}     

Por tanto un individuo es averso al riesgo si se cumple:
\begin{align}
U(W)&> E(U(W+\hat{\epsilon}))=pU(W+h_{1})+(1-p)U(W+h_{2})\nonumber 
\end{align}  
  
\end{frame}


\begin{frame}
    \frametitle{Aversión al riesgo y premio de riesgo}
   Es posible definir el grado de aversión al riesgo de un individuo, para ello definimos el concepto ``premio por riesgo'' que significa la cantidad de un bien que el sujeto está dispuesto a pagar para evitar el riesgo.\\
   Definiendo $\pi$ como el ``premio por riesgo'' de un individuo dado un ``juego justo'' $\hat{\epsilon}$ de manera que el valor máximo que el individuo estaría dispuesto a pagar para evitar el riesgo estaría dado por:
   \begin{align}
   U(W-\pi)&= E(U(W+\hat{\epsilon}))
   \end{align}  
   
    
\end{frame}

\begin{frame}
\frametitle{Aversión al riesgo y premio de riesgo}
Ahora bien, que sucede si suponemos que $\hat{\epsilon}$ es un valor pequeño próximo a cero, por que debemos estudiar sus efectos tomando la aproximación de taylor de la ecuación (1) en torno a $\hat{\epsilon}^{*}=0 $ y $\pi^{*}=0 $.\\
Expandiendo el lado izquierdo de (1) en torno de $\pi^{*}=0 $ tenemos:
\begin{align}
f(x)&\approxeq f(x^{*})+\frac{f'(x^{*})}{1!}(x-x^{*})+\frac{f''(x^{*})}{2!}(x-x^{*})^{2}+\frac{f'''(x^{*})}{3!}(x-x^{*})^{3}+...\nonumber \\
f(x)&\approxeq f(x^{*})+\frac{f'(x^{*})}{1!}(x-x^{*})\nonumber \\
U(W-\pi)&\approxeq U(W-\pi^{*})-U'(W-\pi^{*})(\pi-\pi^{*})\nonumber \\
U(W-\pi)&\approxeq U(W)-\pi U'(W)
\end{align } 

\end{frame}

\begin{frame}
    \frametitle{Aversión al riesgo y premio de riesgo}
   Expandiendo el lado derecho de (1) en torno al punto $\hat{\epsilon}^{*}=0 $ tenemos:
    \begin{align}
   E(U(W+\hat{\epsilon}))&\approxeq E(U(W+\hat{\epsilon}^{*})+U'(W+\hat{\epsilon}^{*})(\hat{\epsilon}-\hat{\epsilon}^{*})+\frac{1}{2}U''(W+\hat{\epsilon}^{*})(\hat{\epsilon}-\hat{\epsilon}^{*})^{2}) \nonumber \\
   E(U(W+\hat{\epsilon}))&\approxeq E(U(W)+\hat{\epsilon}U'(W)+\frac{1}{2}\hat{\epsilon}^{2}U''(W))\nonumber \\
    E(U(W+\hat{\epsilon}))&\approxeq E(U(W))+\frac{1}{2}E(\hat{\epsilon}^{2})E(U''(W))\nonumber \\
     E(U(W+\hat{\epsilon}))&\approxeq E(U(W))+\frac{1}{2}\sigma^{2}E(U''(W))
 \end{align} 
 
 Igualando (2) y (3) para obtener $\pi$:
   \begin{align}
 U(W)-\pi U'(W)&=U(W)+\frac{1}{2}\sigma^{2}U''(W) \nonumber \\
 \pi &=-\frac{1}{2}\sigma^{2}\frac{U''(W) }{U'(W) }\nonumber \\
 \pi &=\frac{1}{2}\sigma^{2} R(W)
 \end{align} 
 
 
  
\end{frame}


\begin{frame}
    \frametitle{Aversión al riesgo y premio de riesgo}
    De (4) tenemos que R(W) es conocido como el Arrow-Prat medida de aversión aboluta al riesgo.\\
    
    \begin{block} {Implicaciones de la ecuación 4}
       \begin{itemize}
           \item El premio por riesgo depende de la incertidumbre del activo en riesgo ($\pi $) y de la aversión absoluta al riesgo R(W). 
           \item Si tanto $U'(W) $ como $\sigma^{2}$ son positivos, para que la prima por riesgo sea positiva la función de utilidad debería ser concava y por tanto  $U''(W) $ ser negativa.
           \item La concavidad de la función de utilidad de un individuo no es suficiente para que la prima por riesgo sea en una cuantía considerable, ello depende de la aversión absoluta al riesgo, puede haber el caso de un individuo con $U''(W) $ muy grande pero que no esté dispuesto a pagar un alto monto de prima por riesgo debido de tratarse de una persona pobre.
       \end{itemize} 
    \end{block}	
    
\end{frame}


\section{Funciones de la banca}
\begin{frame}
	\frametitle{{\normalsize INTRODUCCIÓN} {}}
	\setcounter{equation}{0}
	¿Qué es un banco y qué es lo que hace?\\
    Un banco es una institución cuya actividad {\bf\emph{corriente}} es conceder crédito y recibir depósitos del {\bf\emph{público}}. \\
    Los bancos financian una importante fracción de sus créditos a través de los depósitos del público, ahí la principal explicación de su fragilidad y la justificación de su regulación. Por ello algunos economistas predicen que los bancos serán sustituidos por los fondos mutuos o narrow banking, quienes invierten los depósitos en valores negociados o por otras instituciones financieras quienes conceden crédito a través de la emisión de deuda o acciones.\\
    El término {\bf\emph{público}}, enfatiza que los bancos provee un único servicio al público: Liquidación y medio de pago. Es de notar que el público a diferencia de los inversores institucionales, no tiene los medios para evaluar la solidez de una institución y la calidad de sus activos, por lo que confían en los bancos para proveerse de esos bienes públicos. 
     
    
    	
\end{frame}

\begin{frame}
    \frametitle{{\normalsize FUNCIONES DE UN BANCO} {}}
   Los bancos desarrollan cuatro funciones:
   \begin{itemize}
       \item Ofrecen liquidez y servicios de pagos.
       \item Transforman activos
       \item Administran riesgos
       \item Procesan información y monitorean a los deudores
   \end{itemize}    
\end{frame}

\begin{frame}
    \frametitle{{\normalsize Liquidez y servicios de pagos} {}}
    Dada la existencia de costos de transacción el dinero es el medio de cambio. Hay dos tipos de dinero:
    \begin{itemize}
        \item Dinero mercancía 
        \item Dinero fiduciario
    \end{itemize}
    Esta función de los bancos se puede entender de forma más precisa en dos actividades:
    \begin{itemize}
        \item Cambio de moneda
        \item Servicios de pagos
    \end{itemize}
\end{frame}

\begin{frame}
    \frametitle{{\normalsize Transformación de activos} {}}
    La transformación de activos puede ser a través de 3 perspectivas:
    \begin{itemize}
        \item Conveniencia de denominación (unidad de tamaño)
        \item Transformación de calidad (motivado por: Indivisibilidad de inversión, cuando pequeños depositarios no pueden diversificar su portafolio e información asimétrica a favor de los bancos)
        \item Transformación de madurez (esto implica un riesgo que es mitigado por el crédito interbancario y derivado e instrumentos financieros )
     \end{itemize}
\end{frame}
	
    \begin{frame}
        \frametitle{{\normalsize Administración de riesgos} {}}
     Los riesgos usuales que enfrentan los bancos corresponden a una línea de sus balances. Estos son:
     \begin{itemize}
         \item Riesgo de crédito
         \item Riesgo de tasa de interés
         \item Riesgo de liquidez
     \end{itemize}
     Adicionalmente existe otro riesgo que no se identifica en la hoja de balance de los bancos pero que esta surgiendo en las últimas décadas:
     \begin{itemize}
         \item Riesgo por operaciones fuera de balances
     \end{itemize}
\end{frame}
    
\begin{frame}
    \frametitle{{\normalsize Bancos en el modelo Arrow-Debreu: Esquema de las decisiones económicas de los agentes} {}}
    
    \begin{tikzpicture}
    \node[right] (MF) at (0.0,10) {{\tiny MERCADO FINANCIERO}};
    \node[right] (MF1) at (0.2,9.5){{\tiny $B_{f}+B_{b}=B_{h}$}};
    
    \node[right] (F) at (-3.3,7.5) {{\tiny $Firmas$}};
    \node[left] (F1) at (-3.0,7.3) {{\tiny $Activos$}};
    \node[right] (F2) at (-2.5,7.3){{\tiny $Pasivos$}};
    \node (F11) at (-3.5,7.0)  {{\tiny $Incersión I$}};
    \node (F21) at (-1.7,7.0) {{\tiny Títulos $B_{f}$}};
    \node (F22) at (-1.4,6.75) {{\tiny Créditos $ L^{-}$}};
    
    \node[right] (B) at (0.5,5.0) {{\tiny $Bancos$}};
    \node[left] (B1) at (0.75,4.75) {{\tiny $Activos$}};
    \node[right] (B2) at (1.0,4.75){{\tiny $Pasivos$}};
    \node (B11) at (-0.25,4.5)  {{\tiny  Créditos $ L^{+}$}};
    \node (B21) at (2.25,4.5) {{\tiny Títulos $B_{b}$}};
    \node (B22) at (3.0,4.25) {{\tiny Dpósitos $ D^{-}$}};
    
\node[right] (H) at (3.3,7.5) {{\tiny $Hogares$}};
    \node[left] (H1) at (3.75,7.3) {{\tiny $Activos$}};
\node[right] (H2) at (4.25,7.3){{\tiny $Pasivos$}};
\node[left] (H11) at (3.5,7.0)  {{\tiny Títulos $B_h$}};
\node[left] (H12) at (3.5,6.75)  {{\tiny Depósitos $D^{+}$}};
\node[right] (H21) at (4.5,7.0) {{\tiny Ahorros $S$}};

    
    
    \path[->] (MF1) edge (F21)
    (MF1) edge (B21)
    (B11) edge (F22)
    (H12) edge (B22)
    (H11) edge (MF1);
    \end{tikzpicture}
\end{frame}


\begin{frame}
    \frametitle{{\normalsize Bancos en el modelo Arrow-Debreu} {}}
   \begin{block} {Objetivo}
       Conocer la utilidad de los bancos en el modelo Arrow-Debreu. 	
   \end{block}	
   \begin{block} {Estructura: Hogares}
       {\footnotesize \begin{description}
           \item[Supuesto 1] Los hogares viven en esa economía por 2 períodos y estan dotados de una riqueza inicial: $W_{1}$.
           \item[Supuesto 2] Los hogares seleccionan el perfil temporal de su consumo: $(C_{1}, C_{2})$
           \item[Supuesto 3] Los hogares son los dueños de los bancos y empresas, reciben el periodo 2 los beneficios que las empresas y bancos obtuvieron: $\Pi_{f}$ y $\Pi_{b}$ respectivamente. 
           \item[Supuesto 4] Los hogares deciden como mantener sus ahorro entre tres opciones: Ahorro en el banco en forma  de depósitos $D^{+}$ cuyo rendimiento es $r_{b}$, títulos en bonos de empresas $B_{f} $, cuyo rendimiento es $r_{f} $ y títulos en bancos $B_{h}$ cuyo rendimiento es $r_{b}$. 	
          \end{description}}
     
   \end{block}	
\end{frame}



\begin{frame}
    \frametitle{{\normalsize Bancos en el modelo Arrow-Debreu} {}}
    
    \begin{block} {Estructura: Hogares} 
     Bajo esa estructura el problema de los hogares se resumen en:
    \begin{align}
    \max U(C_{1},C_{2})\nonumber \\
    s.a:\nonumber\\
    C_{1}+B_{h}+D_{h}&=W_{1}\nonumber\\      pC_{2}&=\Pi_{f}+\Pi_{b}+(1+r)B_{h}+(1+r_{b})D_{h}\nonumber
    \end{align}
    \end{block}
      \begin{block} {Implicaciones}
        La solución de la cartera de ahorro de los hogares es interior si y sólo sí se cumple que:
        \begin{align}
        r&=r_{b}
        \end{align} 	
       
    \end{block}	

\end{frame}



\begin{frame}
    \frametitle{{\normalsize Bancos en el modelo Arrow-Debreu} {}}
    \begin{block} {Implicaciones}
        Si la emisión de títulos y el Crédito son sustitutos perfectos se obtiene una solución interior por que se cumple que:
        \begin{align}
        r&=r_{L}
        \end{align} 
     \end{block}
   
\end{frame}


\begin{frame}
    \frametitle{{\normalsize Bancos en el modelo Arrow-Debreu} {}}
    \begin{block} {Estructura: Bancos}
        \begin{description}
            \item[Supuesto 1] Los bancos escogen su oferta de créditos: $L_{b}$, su demanda de depósitos: $D_{b}$ y su emisión de títulos: $B_{b} $.
        \end{description}
        Bajo esa estructura el problema de los bancos es el de máximizar sus beneficios:
        \begin{align}
        \max \Pi_{b}\nonumber \\
        s.a:\nonumber\\
        \Pi_{b}&= r_{L}L_{b}-rB_{b}-r_{d}D_{b}\nonumber\\      
        L&=B_{b}+D_{b}\nonumber
        \end{align}
    \end{block}	
 \end{frame}


\begin{frame}
    \frametitle{{\normalsize Bancos en el modelo Arrow-Debreu} {}}
    \begin{block} {Equilibrio General}
    El equilibrio general esta caracterizado por los vectores: $ (r, r_{L}, r_{D}) $ y tres vectores adicionales de la demanda y oferta de los hogares ($(C_{1}, C_{2}, B_{h}, D_{h}) $), empresas ($(I, B_{f}, L_{f}) $) y bancos ($(L_{b}, B_{b}, D_{b}) $).
    Teniendo en cuenta:
    \begin{itemize}
        \item Cada agente se comporta optimamente.
        \item Cada mercado se clarea:
        \begin{itemize}
            \item $I=S$ (mercados de bienes)
            \item $D_{h}=D_{b} $ (mercado de depósitos)
            \item $L_{f}=L_{b} $ (mercado de créditos)
            \item $B_{H}=B_{f}+B_{b}$ (mercado de bonos)
        \end{itemize}
    \end{itemize}
   {\footnotesize Dada las ecuaciones 1 y 2 está claro que una de las condiciones de equilibrios es que:
   \begin{align}
   r&=r_{L}=r_{b}
   \end{align} }
    \end{block}	
\end{frame}



\begin{frame}
    \frametitle{{\normalsize Bancos en el modelo Arrow-Debreu} {}}
    \begin{block} {Implicaciones}
        \begin{enumerate}
            \item La condición de equilibrio provoca que los beneficios de los bancos sean cero 
            \item Tanto los hogares como las firmas no enfrentan restricciones a un mercado financiero perfecto
           \item El tamaña de los balances bancarios no tienen ningun efecto en otros agentes económicos
           \item El modelo de equilibrio general con mercado financieros completos (el modelo Arrow-Debreu) no pueden ser usado para el estudio del sector bancario (los bancos son redundantes). Hay dos vías para elaborar un modelo util para el análisis de los bancos. Estos son
           \begin{itemize}
               \item El paradigma del mercado incompleto
               \item El recurso de la Organización industrial de los bancos.
           \end{itemize}
        \end{enumerate}
     
    \end{block}	
\end{frame}

\section{Rol de los intermediarios financieros}
\begin{frame}
	\frametitle{{\normalsize LOS INTERMEDIARIOS FINANCIEROS} {}}
	\setcounter{equation}{0}
    Los intermediarios financieros (IF) son todos aquellos agentes especializados en vender y comprar activos financieros.\\
    En terminología de Organización Industrial los IF son análogos a los intermediarios comerciales (retailer). La justificación para su existencia es desde la teorías de Organización Industrial  \textit{la existencia de fricciones en la tecnología de transacción.}\\
    No obstante las actividades de los bancos son más complejas por las razones siguientes:
    \begin{itemize}
        \item Los bancos trabajan con contratos financieros (Créditos y depósitos) que no pueden ser fácilmente disueltos, en oposición a las acciones financieras como valores y bonos los cuales son contratos anónimos en el sentido de que su tenedor es irrelevante lo que lo hace fácil de comercializar.
        \item Los contratos para la emisión de créditos es diferente al contrato para la emisión de un depósito, relacionado al labor de los bancos en transformar activos. 
    \end{itemize} 
    
    
    	
\end{frame}

\begin{frame}
    \frametitle{{\normalsize La existencia de fricciones en la tecnología de transacción: El recurso de los costos de transacción} {}}
    
    En el mundo Arrow-Debreu, los prestatarios y ahorradores pueden diversificar su cartera; no obstante si se introduce dos fricciones: Indivisibilidad y no convexidad en la tecnología de transacción la perfecta diversificación se interrumpe surgiendo la necesidad de los IF.\\
    La vía para introducir los costos de transacción es el de Economías de Escalas, el cual se puede entender como la existencia de un costo fijo por transacción o de manera más general como presencia de retornos crecientes en la tecnología de la transacción. Adicionalmente puede estar relacionado a la seguridad de liquidez, de acuerdo a la ley de los grande números, la mayoría de inversores quieren mantener sus ahorro en títulos no líquidos pero más rentables.\\
    
    
    
    
    
    
    
    
\end{frame}

\begin{frame}
    \frametitle{{\normalsize El modelo de Coalición entre depositantes y seguros de liquidez } {}}
    
    Los servicios que ofrecen las Instituciones Depositarias es el de mantener un pozo liquido al servicio de los hogares que le sirva a ellos de medida de seguridad ante schocks idiosicráticos sobre las necesidades de su consumo.
    
     \begin{block} {Estructura básica del modelo}
        \begin{description}
            \item[Supuesto 1] Consideremos una economía de un sólo bien,  de tres períodos y de un continuo de agentes dotados cada uno de la cantidad de 1 en el período t=0.
            \item[Supuesto 2] El bien debe ser consumido en el período T=1 o T=2 ($C_{1}, C_{2} $)
            \item[Supuesto 3] Hay dos tipos de agentes lo tipo 1 quienes se consumen su dotación en el período 1 y los tipos 2 quienes se consumen su dotación en el período 2.
             \item[Supuesto 4] En el período 1, todos los agentes aprenden a identificar el tipo de agente que hay en la economía.
         \end{description}
        
    \end{block}	
    
    
    
\end{frame}

\begin{frame}
    \frametitle{{\normalsize El modelo de Coalición entre depositantes y seguros de liquidez } {}}
    
   Bajo supuestos básicos tenemos la preferencias de la economía pueden ser representadas por:
   \begin{align}
   U&=\pi_{1}u(C_{1})+\pi_{2}u(C_{2})\nonumber
   \end{align}
   Donde $\pi_{1}$ y  $\pi_{1}$ es la probabilidad de que los agentes sean del tipo 1 o 2, respectivamente.\\
   El bien puede ser almacenado desde el período 1 al 2, o puede ser invertido en una cuenta I, que está en el intervalo $0\leq I \leq 1$ en t=0 y se somete a un cambio tecnológico.\\ 
   El cambio tecnológico permite proveer un $R>1$ de unidades de consumo en T=2, pero sólo  $\iota<1$ en t=1.\\
   
   A continuación se discutirá este modelos en diferentes instituciones, para evaluar en cual de éstas mejora la eficiencia de la economía
   
\end{frame}


\begin{frame}
    \frametitle{{\normalsize El modelo bajo una óptima asignación } {}}
    
    Si tenemos el siguiente problema;
   \begin{align}
   \max_{\{C_{1},C_{2}\}}U&=\pi_{1}u(C_{1})+\pi_{2}u(C_{2})
   \end{align}
   Sujetos a:
   \begin{align}
   \pi_{1}C_{1}&=1-I\\
   \pi_{2}C_{2}&=RI
   \end{align}
   Despejando I de 2 y 3 tenemos:
   \begin{align}
   1-\pi_{1}C_{1}&=I\nonumber\\
   \frac{\pi_{2}C_{2}}{R}&=I\nonumber\\
   \pi_{1}C_{1}+\frac{\pi_{2}C_{2}}{R}&=1
   \end{align}
   Y la condición de equilibrio es:
   \begin{align}
   u^{'}(C_{1}^{*})&=Ru^{'}(C_{2}^{*})
   \end{align}
\end{frame}

\begin{frame}
    \frametitle{{\normalsize El modelo bajo una óptima asignación: Autarquía } {}}
   Este es el caso en el que no existe comercio entre los agentes en la economía. Cada agente escoge independientemente la cantidad de I que va a invertir en la tecnología iliquida, asumiendo que hay perfecta divisibilidad, si el es un agenti tipo 1 entonces su inversión se hará liquida en t=1, representandole:  \\
    \begin{align}
    C_{1}=1-I+\iota I&=1-I(1-\iota)
    \end{align}
   Para el agente tipo 2 ocurre:
   \begin{align}
   C_{2}=1-I+R I&=1+I(R-1)
   \end{align}
   
 Note que el problema del agente sigue siendo 1, pero bajo las restricciones 6 y 7.\\
 Además noten que $C_{1}=1\longleftrightarrow I=0$ y $C_{2}=R\longleftrightarrow I=1$; si no suceden ambos casos la eficiencia no se alcanza por que:
 \begin{align}
 \pi_{1}C_{1}+\frac{\pi_{2}C_{2}}{R}&<1
 \end{align}
 
 
\end{frame}


\begin{frame}
    \frametitle{{\normalsize El modelo bajo una óptima asignación: Economía de mercado } {}}
    Si suponemos ahora que los agentes pueden comerciar entre ellos, es posible agregar un mercado financiero en T=1. En ese mercado los agentes intercambian el bien de consumo por un bono libre de riesgo. Denotando P al precio del bono en t=1, el cual por convención equivaldrá a una unidad del bien de consumo en t=2. Claramente si $p\leq1 $ las personas preferirán acceder al mercado financiero con respecto a guardar el bien.\\
    Por invertir I en t=0, el agente obtiene dependiendo si es del tipo 1 o tipo 2:
      \begin{align}
      C_{1}&=1-I+PRI\\
      C_{2}=\frac{1-I}{P}+RI&=\frac{1}{P}\left( 1-I+PRI\right) 
      \end{align}
    Es de notar que de acuerdo a las ecuaciones 8 y 9, tenemos a siguiente identidad:
     \begin{align}
     C_{2}&=\frac{C_{1}}{P}
    \end{align}

 \end{frame}

\begin{frame}
\frametitle{{\normalsize El modelo bajo una óptima asignación: Economía de mercado } {}}
Dado que I puede ser escogido libremente por los agentes el único  posible precio de equilibrio de interior es:
\begin{align}
P&=\frac{C_{1}}{R}
\end{align}
\end{frame}


\begin{frame}
    \frametitle{{\normalsize El modelo bajo una óptima asignación: Economía de mercado } {}}
    En otro caso ocurre un exceso de oferta o un exceso de demanda de bonos:
     \[
    I= \left\{ \begin{array}{lcl}
    1 & \mbox{ si } & P>\frac{1}{R} \\
    & & \\
    0 & \mbox{ si } & P<\frac{1}{R}
    \end{array}
    \right.
    \]
    
    La asignación de equilibrio de mercado es $C_{1}^{M}=1$ y $C_{2}^{M}=R$. Y los correspondientes niveles de inversión es $ I^{M}=\pi_{2}$.\\
    Es de notar que la asignación de mercado, es Pareto-dominante respecto a la asignación de autorquía, aunque todavía se aleja de la asignación pareto-óptima dado que no satisface:
    \begin{align}
    u^{'}(C_{1}^{M})&=Ru^{'}(C_{2}^{M})\nonumber \\
    u^{'}(1)&=Ru^{'}(R)\nonumber
    \end{align}
    Y es que si $Cu^{'}(C)$ es decreciente, en el caso de $R>1$ tenemos que $u^{'}(1)>Ru^{'}(R)$ y la asignación de mercado mejoraría aumentando el $C_{1}^{M}$ y disminuyendo $C_{2}^{M}$: $C_{1}^{M}=1<C_{1}^{*} $ y $C_{2}^{M}=R>C_{2}^{*} $.
    \end{frame}




\begin{frame}
    \frametitle{{\normalsize El modelo bajo una óptima asignación: Economía de mercado } {}}
    
     \begin{block} {Implicaciones}
       Es de notar que la economía de mercado no provee una seguridad ante choques de liquidez, ello es debido a que estos choques no son públicamente observables y por tanto, un seguro contingente para ellos no puede ser comercializado.\\
       La razón es que la asignación de mercado no es Pareto-óptima, ya que el mercado no es completo, debido a que el estado de la economía (la lista de agentes tipo 1) no es observable por nadie, el único mercado financiero introducido en el modelo (el de lo bonos) no es suficiente para tener una eficiente diversificación del riesgo.
        
    \end{block}	
    
\end{frame}

\begin{frame}
    \frametitle{{\normalsize El modelo con intermediarios financieros } {}}
    
    La asignación Pareto-óptima dada por el par ($C_{1}^{*}, C_{2}^{*}$) que haga cumplir la condición:
    \begin{align}
    u^{'}(C_{1}^{*})&=Ru^{'}(C_{2}^{*})\nonumber \\
    \end{align}
    Puede ser implementada al introducir un IF que ofrezca un contrato de depósitos en los siguientes términos:
    \begin{itemize}
        \item Por un depósito de una unidad del bien en t=0.
        \item El agente que haga el depósito podrá recibir $C_{1}^{*}$ en t=1, ó $ C_{2}^{*}$ en t=2, como él lo decida.
     \end{itemize}
 Para cumplir esa obligación el IF deberá almacenar $\pi_{1}C_{1}^{*} $ e invertir $I=1-\pi_{1}C_{1}^{*}$ en tecnología ilíquida.
\end{frame}

\begin{frame}
    \frametitle{{\normalsize El modelo con intermediarios financieros } {}}
    
    \begin{block} {Implicaciones}
       \begin{enumerate}
           \item En una economía en la cuál los agentes sujetos individuales independientes de choques de liquidez, la asignación de mercado puede mejorar por la provisión de un contrato de depósito ofrecido por un IF.
           \item El banco que ofrezca servicios de depósitos no requerirá de capital; pues el riesgo de liquidez está perfectamente diversificado y el crédito no trae un riesgo (ver la hoja de balance del banco).
           \item Un supuesto crucial es que ningún individuo del tipo 2 retirara sus depósitos en t=1, este supuesto no es irrelevante, dado que $u^{'}$ es decreciente y que $R>1$, tenemos que $C_{1}^{*}<C_{2}^{*}$ por tanto si el sujeto tipo 2 se retira en t=1 y almacena él mismo el bien, éste será menor. No obstante si este supuesto puede existir un equilibrio pareto-dominante.
        \newcounter{enumTemp}
        \setcounter{enumTemp}{\theenumi}   
       \end{enumerate}
      
    \end{block}	
    
\end{frame}

\begin{frame}
    \frametitle{{\normalsize El modelo con intermediarios financieros } {}}
    
    \begin{block} {Implicaciones}
        \begin{enumerate}
            \setcounter{enumi}{\theenumTemp}
            \item Los IF no pueden coexistir con un mercado de bonos; si mantenemos la existencia del mercado de bonos en t=1, el precio de equilibrio necesariamente será $P=\frac{1}{R}$ y la asignación óptima ($C_{1}^{*}, C_{2}^{*}$)  ya no sería un equilibrio pues:
            \begin{align}
            RC_{1}^{*}>R>C_{2}^{2}\nonumber 
            \end{align}
            Lo que significaría que el consumidor tipo 2, no tendría razón o incentivos para retirarse temprano y comprar bonos. Siendo está una debilidad del modelo. 
        \end{enumerate}
        
    \end{block}	
    
\end{frame}

\section{Selección adversa, costo del capital y coalición de prestatarios}
\include{4_CLASE}
\section{Delegando el monitoreo}
\begin{frame}
	\frametitle{{\normalsize INTRODUCCIÓN} {}}
	\setcounter{equation}{0}
En el contexto de información asimétrica el monitoreo sería una forma para mejorar la eficiencia.\\
Siguiendo a Hellwig (1991) el monitoreo en el sentido amplio significa:
\begin{itemize}
    \item Filtrar proyectos en contexto de selección adversa (\cite{Broecker1990}) 
    \item Previniendo el comportamiento oportunistas de las firmas (moral hazard) \cite{Tirole1997} 
    \item Castigando o auditando a una firma o banco que incumpla una obligación contractual \cite{Diamond1984}
\end{itemize}    
  
La idea central es que los bancos tienen una \textbf{ventaja comparativa} en el monitoreo de esa actividad. Esta ventaja puede deberse a:
\begin{itemize}
    \item Economía de escala en el monitoreo\\
    \item Pequeña capacidad de los inversionistas en relación al monto de inversión del proyecto
    \item Bajo costo de delegar. El costo de monitoreo a si misma de las IF, debe ser menor que el superávit ganado por aprovechar la economía de escala en el monitoreo de los proyectos.
\end{itemize} 

\end{frame}

\begin{frame}
    \frametitle{{\normalsize Intermediación Financiera como delegación del monitoreo} {}}
    
    \begin{block} {Estructura básica del modelo}
        \begin{description}
            \item[Supuesto 1] Consideramos una economía con $n$ firmas.
            \item[Supuesto 2] Cada una de las firmas cuentan con un proyecto riesgoso, el cual requiere una inversión de costo fijo normalizado en 1. 
            \item[Supuesto 3] El retorno de cada firma es idéntico e independientemente distribuido.  
            \item[Supuesto 4]  Las firmas son neutrales al riesgo.
            \item[Supuesto 5] Cada flujo $\hat{y}$ que la firma obtiene por su inversión es inobservable para los inversionistas.
            \end{description}
        
    \end{block}	
    
\end{frame}





\begin{frame}
    \frametitle{{\normalsize Intermediación Financiera como delegación del monitoreo} {}}
    
     \begin{block} {Estructura básica del modelo}
        \begin{description}
            \item[Supuesto 6] Pagando una cuota $K$ por el costo de monitoreo los inversionistas pueden observar el flujo de efectivo del proyecto y hacer cumplir el contrato del crédito pagando $\hat{y} $. Por tanto los inversionistas tienen una rentabilidad de:
            \begin{align}
            E(\hat{y})>1+r+K
            \end{align}
            Donde r es la tasa libre de riesgo.
            \item[Supuesto 7] Suponga que cada inversionista sólo puede financiar una fracción $\frac{1}{m}$ de cada proyecto. Para financiar todos los proyectos se requieren $n*m$ inversionistas. Así que el financiar esos proyectos implicaría un costo de monitoreo total de $ n*m*K$
           \end{description}
       
      
       
     \end{block}	
 

 

\end{frame}


\begin{frame}
    \frametitle{{\normalsize Intermediación Financiera como delegación del monitoreo} {}}
   
    En este contexto si los bancos surgen, cada inversor debe de pagarle el costo de monitoreo al banco y el banco por su parte pagar el costo de monitoreo de cada firma. Es decir que el costo total de monitoreo sería:
    \begin{align}
    n*K+n*m*K
    \end{align}
     \begin{block} {Conclusión}
     La aparición de los bancos, provoca un costo adicional relacionado al nuevo eslabón en el proceso de monitoreo, agravando la ineficiencia. \\
     La idea de \cite{Diamond1984} es que los incentivos de los bancos sea provisto por una alternativa tecnológica: La auditoría.   
    \end{block}
    
    
   
\end{frame}

\begin{frame}
    \frametitle{{\normalsize Tecnología de la auditoría} {}}
    
    Incentivar a los bancos a no responder a un mismo tiempo a todos los depósitos dependerá de que éstos no cierren o caigan en bancarrota. \\
    Los bancos prometen un tasa fija por cada depósito ($r_{D}$)  además el banco es auditado sólo si el retorno de los activos del banco no es suficiente para cumplir su promesa.
    El costo por la auditoría tiene la ventaja de ser fija e independiente del número de inversores y está dada por:
    \begin{align}
    C_{n}&=n\gamma Pr(\hat{y}_{1}+\hat{y}_{2}+\cdot+\hat{y}_{n}<(1+r_{D})n),
    \end{align}
    Donde $\gamma$ es el costo de unidad auditada. Asumimos que el costo de auditoría es proporcional al volumen de activos y que se cumple $K<C$, lo que significa que si una firma tiene un proyecto capaz de financiarse por un sólo inversionista  es eficiente la opción del monitoreo directo. 
    
\end{frame}


\begin{frame}
    \frametitle{{\normalsize Tecnología de la auditoría} {}}
    
    Si un banco surge, debe de elegir su propia tecnología de monitoreo tanto para sus créditos como para su depósitos. En el caso de los crédito hay dos tecnologías de monitoreo siendo sus costos respectivos:
    \begin{enumerate}
        \item n*K en el caso en que monitoree directamente cada proyecto
        \item n*C en el caso tenga que pagar el costo de soportar una auditoría directa.
    \end{enumerate}
    Entre ambos casos el menor costo es el del monitoreo directo. Por tanto el banco es un monitor delegado sobre las firmas en nombre de los inversores. \\
    El nivel de equilibrio de $r_{b} $ y el costo esperado de la auditoria dependerá de n. Estos son determinados implícitamente por:
    \begin{align}
    E(\min(\frac{1}{n}\sum_{i=1}^{n}\hat{y}_{i}-K,\; \; 1+r_{D}^{n}))&=1+r \\
    C_{n}&=n\gamma Pr(\frac{1}{n}\sum_{i=1}^{n}\hat{y}_{i}-K<(1+r_{D}^{n}))
    \end{align}
    
   
\end{frame}

\begin{frame}
\frametitle{{\normalsize Tecnología de la auditoría} {}}

Noten que $\frac{1}{n}\sum_{i=1}^{n}\hat{y}_{i}-K $ es el retorno neto de los activos bancarios. Por tanto la situación de delegar el monitoreo al banco será más eficiente que el monitoreo lo haga cada inversionista, si y sólo si se cumple:
\begin{align}
nK+C_{n}<n*m*K
\end{align}
\begin{block} {Implicaciones 1. (\cite{Diamond1984})}
	Si el monitoreo es eficiente $(K<C)$, los inversores son pequeños $(m>1) $ y el beneficio de la inversión es $ E(\hat{y})>1+r+k$. El monitoreo por los bancos domina sobre el monitoreo por los inversionistas en la medida n sea más grande (diversificación). 
\end{block}
\end{frame}


\begin{frame}
    \frametitle{{\normalsize Tecnología de la auditoría} {}}
    
   Comprobación. Dado:
    \begin{align}
   nK+C_{n}<n*m*K \nonumber\\
   K+\frac{C_{n}}{n}<m*K\nonumber\\
   \lim_{n\rightarrow\propto}\frac{C_{n}}{n}&\approx 0
   \end{align}
    Además por la ley de los grandes números: $\frac{1}{n}\sum_{i=1}^{n}\hat{y}_{i} $ converge a $E(hat{y})$ y dado que $ E(\hat{y})>1+r+k$, entonces por 4, tenemos que:
     \begin{align}
    \lim_{n\rightarrow\propto}r_{D}^{n}&\approx r
    \end{align}
    Los depósitos son una inversión sin riesgo.
   
\end{frame}

\begin{frame}
    \frametitle{{\normalsize Tecnología de la auditoría} {}}
    En conclusión se debe cumplir que:
    \begin{align}
    \lim_{n\rightarrow\propto}\frac{C_{n}}{n}&=\lim_{n\rightarrow\propto}Pr(\frac{1}{n}\sum_{i=1}^{n}\hat{y}_{i}-k<1+r_{D}^{n})=\lim_{n\rightarrow\propto}Pr(E(\hat{y})-k<1+r)\approx 0
    \end{align}
\end{frame}
\section{Deuda de mercado vs. deuda de bancos}
\begin{frame}
	\frametitle{{\normalsize INTRODUCCIÓN} {}}
		\setcounter{equation}{0}
El crédito bancario tiene la cualidad de la \textit{Unicidad} concepto dado por \cite{James1987}, quien notó que el mercado reacciona de manera positiva ante una firma cuando se da cuenta que sus proyectos son financiados en alguna cuota por los bancos.\\

Sin embargo, en los últimos años ha crecido el financiamiento de las firmas de manera directa (proceso de des-intermediación), especialmente entre firmas grandes.\\
En la práctica, la deuda directa es menos costosa que la indirecta, dado que el crédito sólo es buscado por aquellas firmas que no pueden acceder al mercado de deuda directa.\\
Por tanto el objetivo de explicar la coexistencia de estas dos modalidades de financiamiento se ha basado en el problema de \textit{moral hazard}, la idea es evitar aquellas firmas que no cuenten con el suficiente activo para obtener el financiamiento de manera directa.

\end{frame}

\begin{frame}
    \frametitle{{\normalsize UN SIMPLE MODELO DEL MERCADO DE CRÉDITO CON MORAL HAZARD} {}}
    
    \begin{block} {Estructura básica del modelo}
        \begin{description}
            \item[Supuesto 1] Las firmas buscan inversionistas para proyectos cuya inversión sea de costos fijos normalizados en 1.
            \item[Supuesto 2] La tasa libre de riesgo está normalizada a 0 ($r=0$). 
            \item[Supuesto 3] Las firmas tienen que escoger entre los siguientes proyectos:
            \begin{enumerate}
               \item \textit{Tecnología buena.} la cual produce un suceso G con una probabilidad de $\pi_{G} $ y cero en el otro caso.
               \item \textit{Tecnología mala.} la cual produce un suceso B con una probabilidad de $\pi_{B} $ y cero en el otro caso.
            \end{enumerate}
            \item[Supuesto 4] Sólo los buenos proyectos tienen un valor presente neto positivo. Es de notar que: $\pi_{G}G>1>\pi_{B}B $, pero $B>G\rightarrow \pi_{G}>\pi_{B} $; esto último significa que a mayor rentabilidad menor probabilidad de ocurrencia. 
           
             
            \end{description}
        
    \end{block}	
    
\end{frame}


\begin{frame}
    \frametitle{{\normalsize UN SIMPLE MODELO DEL MERCADO DE CRÉDITO CON MORAL HAZARD} {}}
    
    \begin{block} {Estructura básica del modelo}
        \begin{description}
             \item[Supuesto 5] Asuma que el suceso de la inversión es verificable por externos; pero no la elección de tecnología de las firmas ni su retorno. Por tanto la firma puede hacer una promesa de pago de un monto fijo R sólo en caso se da el suceso de la inversión.
            \item[Supuesto 6] Las firmas no tienen otra fuente de recursos, así que no paga la promesa en caso la inversión falla.
            
        \end{description}
          
    \end{block}	

 
\end{frame}

\begin{frame}
\frametitle{{\normalsize UN SIMPLE MODELO DEL MERCADO DE CRÉDITO CON MORAL HAZARD} {}}

Es de notar que el valor R del endeudamiento de las firmas determina la elección de la tecnología. Adicionalmente en ausencia de monitoreo, la firma escoge la tecnología buena si y sólo si esta produce un beneficio esperado alto:
\begin{align}
\pi_{G}(G-R)&>\pi_{B}(B-R) \\
ó \nonumber \\
R<R_{C}&=\frac{\pi_{G}G-\pi_{B}B}{\pi_{G}-\pi_{B}}    
\end{align}
Donde $R_{C}$ denota el valor crítico de la deuda nominal con la cual si R es mayor que $R_{C}$ la firma escoge la mala tecnología. 
\end{frame}

\begin{frame}
    \frametitle{{\normalsize UN SIMPLE MODELO DEL MERCADO DE CRÉDITO CON MORAL HAZARD} {}}
 Desde el punto de vista de la firma que toma el crédito, la probabilidad de pagar dicho compromiso depende de R:
 
 \[\pi(R)=\left\{ \begin{array}{rcl}
 \pi(G) & \mbox{si} & R\leq R_{C}\\
 & & \\
 \pi(B) & \mbox{si} & R > R_{C}\\
 \end{array}
 \right. \] 

En ausencia de monitoreo, el equilibrio competitivo del mercado de crédito se obtiene un R tal que:
  \begin{align}
  \pi(R)R=1
  \end{align}    
Ya que $\pi_{B}R<1$ para todo $R\leq B$. El equilibrio es solo posible cuando la tecnología buena es implementada, implicando que  $R<R_{C}$ y $\pi_{G}R_{C}\geq 1$ esto sólo se satisface si el moral hazard no es muy importante.\\
Si  $\pi_{G}R_{C}<1$, el equilibrio implica una ausencia de comercio y por tanto el mercado de crédito colapsa, dado que los proyectos buenos no son financiado y los malos tienen un valor presente neto esperado negativo.

\end{frame}

\begin{frame}
\frametitle{{\normalsize UN SIMPLE MODELO DEL MERCADO DE CRÉDITO CON MORAL HAZARD} {}}
Introduciendo una tecnología de monitoreo:\\
Con un costo C un banco puede prevenir a las firmas a utilizar la mala tecnología- Suponiendo competencia perfecta entre bancos, el valor nominal de los créditos bancarios en equilibrio (denotado por $R_{m}$ donde m denota al monitor) está dada por la condición de quiebre:
\begin{align}
\pi_{G}R_{m}&=1+C
\end{align}   
\end{frame}

\begin{frame}
\frametitle{{\normalsize UN SIMPLE MODELO DEL MERCADO DE CRÉDITO CON MORAL HAZARD} {}}
   
Para que el crédito bancario se encuentre en equilibrio, dos condiciones son necesarias:
\begin{enumerate}
	\item El reembolso nominal del crédito bancario en equilibrio tiene que ser menor que el retorno G que reciben las firmas. Ello es equivalente a:
	\begin{align}
	\pi_{G}G-1&>C
	\end{align} 
	Es decir que el costo de monitoreo tiene que ser menor al valor presente neto del proyecto bueno 
	\item El financiamiento directo, el cual es menos costoso, tiene que ser imposible
	\begin{align}
	\pi_{G}R_{C}&<1
	\end{align} 
\end{enumerate}

\end{frame}


\begin{frame}
    \frametitle{{\normalsize UN SIMPLE MODELO DEL MERCADO DE CRÉDITO CON MORAL HAZARD} {}}
   Así el crédito bancario aparece en equilibrio para valores intermedios en probabilidad
    \begin{align}
    \pi_{G}(\pi_{G} \in \left[\frac{1+C}{G},\; \frac{1}{R_{C}}\right] )\nonumber
    \end{align} 
    el cual es un intervalo que no es vacío.
    
\end{frame}

\begin{frame}
\frametitle{{\normalsize UN SIMPLE MODELO DEL MERCADO DE CRÉDITO CON MORAL HAZARD} {}}
Por lo que se debe de establecer lo siguiente:
\begin{block} {Resultado} 
	Suponga que el costo de monitores es los suficientemente pequeño que $\frac{1}{R_{C}}>\frac{1+C}{G} $. Entonces hay tres posibles regímenes de mercado de crédito de equilibrio:
	\begin{enumerate}
		\item Si $\pi_{G}>\frac{1}{R_{C}}$ suceso altamente probable, las firmas piden prestado a los bancos a una tasa: $R_{1}=\frac{1}{\pi_{G}}$.
		\item  Si $\pi_{G}\in \left[\frac{1+C}{G},\; \frac{1}{R_{C}}\right] $ suceso medianamente probable, las firmas piden prestado a los bancos a una tasa: $R_{2}=\frac{1+C}{\pi_{G}}$. 
		\item  Si $\pi_{G} < \frac{1+C}{G} $ suceso poco probable, el mercado de crédito colapsa.
	\end{enumerate}    
\end{block}    

\end{frame}


\section{Monitoreo y reputación}
\begin{frame}
	\frametitle{{\normalsize INTRODUCCIÓN} {}}
	\setcounter{equation}{0}
Basados en  \cite{Diamond1991} se extiende el modelo del mercado de crédito con moral hazard incorporando más períodos (es decir dándole dinámica, t=0,1). El objetivo es que las firmas puedan generar cierta reputación para financiarse directamente y no a través de créditos que es más caro.

 

\end{frame}

\begin{frame}
    \frametitle{{\normalsize MONITOREO Y REPUTACIÓN} {}}
    
    \begin{block} {Estructura básica del modelo}
        \begin{description}
              \item[Supuesto 1]  Las firmas buscan inversionistas para proyectos cuya inversión sea de costos fijos normalizados en 1.
            \item[Supuesto 2] La tasa libre de riesgo está normalizada a 0 ($r=0$). 
            \item[Supuesto 3] Las firmas tienen que escoger entre los siguientes proyectos:
            \begin{enumerate}
               \item \textit{Tecnología buena.} la cual produce un suceso G con una probabilidad de $\pi_{G} $ y cero en el otro caso.
               \item \textit{Tecnología mala.} la cual produce un suceso B con una probabilidad de $\pi_{B} $ y cero en el otro caso.
            \end{enumerate}
            \item[Supuesto 4] Sólo los buenos proyectos tienen un valor presente neto positivo. Es de notar que: $\pi_{G}G>1>\pi_{B}B $, pero $B>G\rightarrow \pi_{G}>\pi_{B} $; esto último significa que a mayor rentabilidad menor probabilidad de ocurrencia. 
           
             
            \end{description}
        
    \end{block}	
    
\end{frame}


\begin{frame}
    \frametitle{{\normalsize MONITOREO Y REPUTACIÓN} {}}
    
    \begin{block} {Estructura básica del modelo}
        \begin{description}
             \item[Supuesto 5] Asuma que el suceso de la inversión es verificable por externos; pero no la elección de tecnología de las firmas ni su retorno. Por tanto la firma puede hacer una promesa de pago de un monto fijo R sólo en caso se da el suceso de la inversión.
            \item[Supuesto 6] Las firmas no tienen otra fuente de recursos, así que no paga la promesa en caso la inversión falla.
           \item[Supuesto 7] Con el objeto de introducir el concepto de reputación, supondremos firmas heterogéneas, sólo una fracción f de las firmas tienen acceso a los dos tipos de tecnologías. El resto sólo acceden a la tecnología mala y el monitoreo de los bancos no los afecta. 
        \end{description}
          
    \end{block}	

   
\end{frame}



\begin{frame}
\frametitle{{\normalsize MONITOREO Y REPUTACIÓN} {}}	

Bajo algunas condiciones de los parámetros del modelo, el equilibrio del mercado de crédito debería de ser tal que:
\begin{enumerate}
	\item En t=0, todas las firmas son financiadas por los bancos
	\item En t=1. las firmas con un buen suceso en t=0, pueden financiarse de manera directa, el resto se financia por los bancos.
	\item Los bancos monitorean a las firmas que financian.
\end{enumerate} 
\end{frame}


\begin{frame}
    \frametitle{{\normalsize MONITOREO Y REPUTACIÓN} {}}

 Una firma con un buen suceso en t=0 podrían emitir deuda directa si se cumple que:
 \begin{align}
 \pi_{s}&>\frac{1}{R_{C}}
 \end{align}
 Donde $\pi_{s} $ es la probabilidad de reponer el financiamiento en la fecha 2, condicional al suceso en la fecha 0. Aplicando la formula de Bayes, tenemos:
 \begin{align}
 \pi_{s}&=\frac{P(\mbox{suceso en t=0 y t=1})}{P(\mbox{suceso en t=0})}=\frac{f\pi_{G}^{2}+(1-f)\pi_{B}^{2}}{f\pi_{G}+(1-f)\pi_{B}}
 \end{align}
 Si se satisface 1, la firmas con suceso pueden emitir deuda directa
  a una tasa $R_{s}\frac{1}{\pi_{s}} $. La probabilidad de que aquellas firmas con mal suceso en t=0, tengan un buen sueceso en t=1 es:
 \begin{align}
 \pi_{m}&=\frac{f\pi_{G}(1-\pi_{G})+(1-f)\pi_{B}(1-\pi_{B})}{f(1-\pi_{G})+(1-f)(1-\pi_{B})}
 \end{align}
  
 \end{frame}

\begin{frame}
    \frametitle{{\normalsize MONITOREO Y REPUTACIÓN} {}}
Considerando el el nivel critico de tasa, tenemos:
  \begin{align}
  \frac{1+C}{G}&<\pi_{m}<\frac{1}{R_{C}}
  \end{align} 
Las firmas con mal suceso deben prestarle a los bancos a una tasa:
 \begin{align}
 R_{m}&=\frac{1+C}{\pi_{m}}
 \end{align}   
En orden de completar el modelo, es necesario solo establecer que en t=0, para adecuados valores de los parámetros, todas las firmas (dado que no hay forma de distinguirlas) se financia a través de los bancos.\\
Si $\pi_{0}$ denota la probabilidad incondicional del suceso en t=0 (llamada la estrategia de la firma en seleccionar el proyecto bueno dado que son monitoreadas)
\begin{align}
\pi_{0}&=f\pi_{G}+(1-f)\pi_{B}
\end{align}   

\end{frame}

\begin{frame}
    \frametitle{{\normalsize MONITOREO Y REPUTACIÓN} {}}
    La noción de reputación se construye desde el hecho $\pi_{m}<\pi_{0}<\pi_{s} $, esto es que la probabilidad de reponer la deuda pra la firma es inicialmente $\pi_{0}$, pero esta se incrementa si las firmas tienen un buen suceso ($\pi_{s}$) y decrece en el otro caso ($\pi_{m}$). Esto se debe a que el nivel critico de deuda ($R_{C}^{0} $) en t=0 es alto con respecto al caso estático. Adicionalmente las firmas saben que si ellos tienen un buen suceso en t=0, ellos pueden obtener financiamiento barato ($R_{s} $ en vez de  $ R_{m}$) en t=1. Si $\delta<1$ denota el factor de descuento, el nivel crítico de deuda el cual estratégicamente la firma escoge  los malos proyectos en t=0 (denotado por $R_{C}^{0}$) es ahora definido por:
  \begin{align}
  \pi_{B}\left[B-R_{C}^{0}+\delta\pi_{G}(G-R_{s}) \right]+(1-\pi_{B})\delta \pi_{G}(G-R_{m})&= \nonumber \\
  \pi_{G}\left[G-R_{C}^{0}+\delta\pi_{G}(G-R_{S}) \right]+(1-\pi_{G})\delta \pi_{G}(G-R_{m}) 
  \end{align}  
  
\end{frame}


\begin{frame}
\frametitle{{\normalsize MONITOREO Y REPUTACIÓN} {}}
 
El lado izquierdo de esta identidad es la suma descontada de los beneficios esperados de las firmas que escogen la mala tecnología en t=0. Note que en t=1, podrían escogerse la tecnología buena ya sea emitiendo deuda directamente al mercado a una tasa $R_{S} $ o pidiendo crédito a los bancos a una tasa $R_{m} $. El lado derecho representa la suma descontada de los beneficios esperados por las firmas que escogen el proyecto bueno en t=0 (así también en t=1). Solucionando R tenemos:

\begin{align}
R_{C}^{0}&=\frac{\pi_{G}G-\pi_{B}B}{\pi_{G}-\pi_{B}}+\delta \pi_{G}(G-R_{s})-\delta \pi_{G}(G-R_{m})\nonumber \\
R_{C}^{0}&=R_{C}+\delta \pi_{G}(R_{m}-R_{s})  
\end{align}

\end{frame}

\begin{frame}
    \frametitle{{\normalsize MONITOREO Y REPUTACIÓN} {}}
   
   
  \begin{block} {Resultado}
  Bajo el siguiente supuesto:
    \begin{align}
    \pi_{0}&\leq \frac{1}{R_{C}^{0}}\;\;\;\pi_{s}> \frac{1}{R_{C}^{0}}\;\wedge\;\frac{1}{R_{C}^{0}}>\pi_{m}>\frac{1+C}{G}
    \end{align}  
    El equilibrio se caracteriza por:
    \begin{enumerate}
        \item En t=0, todas las firmas se financia por los bancos a $R_{0}=\frac{1+C}{\pi_{0}}$
        \item En t=1. las firmas con buen suceso en t=0 se financia de manera directa a 
        $R_{s}=\frac{1}{\pi_{s}}$. Mientras el resto de firmas se financian con crédito a una tasa $R_{m}=\frac{1+C}{\pi_{m}}$, lo cual es más alto que $R_{0}$.
        
    \end{enumerate}  
  \end{block}	
   
\end{frame}


\begin{frame}
    \frametitle{{\normalsize MONITOREO Y REPUTACIÓN} {}}
      
    \begin{block} {Conclusión}
        El modelo anterior permite capturar múltiples características del mercado de crédito:
       \begin{enumerate}
           \item Firmas con buena reputación pueden emitir deuda de forma directa.
           \item Las Firmas con malos sucesos pagan más altas tasas que una firma nueva ($ R_{m}>R_{0}$).
           \item El moral hazard es particularmente aliviado por el efecto de reputación  ($R_{c}^{0}>R_{C}$)
       \end{enumerate}
    \end{block}	
    
\end{frame}


\section{Monitoreo y capital}
\begin{frame}
	\frametitle{{\normalsize INTRODUCCIÓN} {}}
		\setcounter{equation}{0}
Basados en  \cite{Tirole1997} se considera un modelo que captura la noción de sustitutabilidad entre capital y monitoreo, tanto a nivel de las firmas como a nivel de los bancos. Ello se obtiene delegando el monitoreo sin una completa diversificación. \\
El moral hazard en el nivel de bancos es solucionado por el capital bancario. A la vez se asume una perfecta correlación entre los proyectos financiados por los bancos.\\



 

\end{frame}

\begin{frame}
    \frametitle{{\normalsize MONITOREO Y CAPITAL} {}}
    
    \begin{block} {Estructura básica del modelo}
        \begin{description}
            \item[Supuesto 1]  Una economía con tres tipos de agentes:
            \begin{enumerate}
                \item Las firmas (prestadores) representado por el índice f.
                \item El monitor (bancos) representado por el índice m.
                \item El inversor desinformado (depositador) representado por el índice u.
            \end{enumerate}  
            \item[Supuesto 2] Cada proyecto de inversión tiene un costo I y un retorno y el cual es verificable en caso suceda. 
            \item[Supuesto 3] Hay dos tipos de proyectos:
            \begin{enumerate}
               \item \textit{Buen proyecto} con alta probabilidad de suceso $p_{H}$.
               \item \textit{Mal proyecto} con baja probabilidad de suceso $p_{L}$ ($\Delta p= p_{H}-p_{L}$). 
            \end{enumerate}
            \item[Supuesto 4] Los malos proyectos dan un beneficio privado (B) a los prestadores siendo ése la fuente del moral hazard.
            
           
             
            \end{description}
        
    \end{block}	
    
\end{frame}


\begin{frame}
    \frametitle{{\normalsize MONITOREO Y CAPITAL} {}}
    
    \begin{block} {Estructura básica del modelo}
        \begin{description}
             \item[Supuesto 5] Ser una firma monitoreada implica una reducción de los beneficios desde B hasta b por el costo del monitoreo C.
            \item[Supuesto 6] Los inversores son neutral al riesgo, no están informados, no tienen acceso a monitorear las firmas y tienen acceso a una alternativa de inversión que les redime un retorno bruto esperado de $1+r$.
           \item[Supuesto 7] Los buenos proyectos tienen un valor presente neto esperado positivo, sólo si los beneficios privados de las firmas son incluidos: $p_{H}y>1+r>p_{L}y+B $
           
        \end{description}
          
    \end{block}	

   \end{frame}

\begin{frame}
\frametitle{{\normalsize MONITOREO Y CAPITAL} {}}

\begin{block} {Estructura básica del modelo}
	\begin{description}
		\item[Supuesto 8] Las firmas difieren entre ellas sólo por su nivel de capital A, el cual es observable. La distribución del capital es un continuo entre la población de firmas y está dado por la función acumulativa $G()$.
		\item[Supuesto 9] El capital de los bancos es exógeno. Dado esto se asumirá que los activo de los bancos están perfectamente correlacionados con un único parámetro relevante que es el Capital total de la industria bancaria $K_{m}$ y que determina a la vez la capacidad de crédito de la industria.  
	\end{description}
	
\end{block}	

\end{frame}


\begin{frame}
    \frametitle{{\normalsize OPCIONES DE FINANCIAMIENTO: FINANCIAMIENTO DIRECTO} {}}

 Una firma puede financiarse directamente de los inversores desinformados prometiendoles un retorno $R_{u}$ en caso el proyecto suceda  a cambio de I. Si las firmas seleccionaran siempre el proyecto bueno tendríamos el borde superior de $R_{u}$:
 \begin{align}
 p_{H}(y-R_{u})&\geq p_{L}(y-R_{u})+B\leftrightarrow R_{u}\leq y-\frac{B}{\Delta p}
 \end{align}
 Las restricciones de la racionalidad individual de los inversores dado que no son uniformes implica un borde superior de $I_{u}$:
 \begin{align}
 p_{H}R_{u}&\geq (1+r)I_{u}\rightarrow I_{u}\leq \frac{p_{H}}{1+r}(y-\frac{B}{\Delta p})
 \end{align}
 Así que el proyecto puede ser financiado sólo si la firma tiene suficiente capital:
 \begin{align}
 A+I_{u}&\geq I \rightarrow A \geq \hat{A}(r) 
 \end{align}
 Donde se define $\hat{A}(r)$ como $I-\frac{p_{H}}{1+r}(y-\frac{B}{\Delta p})$
 
 \end{frame}


\begin{frame}
    \frametitle{{\normalsize OPCIONES DE FINANCIAMIENTO: FINANCIAMIENTO INDIRECTO} {}}
    
   Si la firma no tiene suficiente capital para emitir deuda de manera directa, puede financiarse en $I_{m}$ de los bancos a los cuales les promete un retorno de $R_{m}$ en caso se de el suceso del proyecto; o bien podría financiarse de manera directa de $I_{u}$ de los inversores desinformados y prometer $R_{u}$ si el proyecto sucede. Por tanto el incentivo de comparabilidad entre restricciones de la firma viene dado por:
    \begin{align}
   p_{H}(y-R_{u}-R_{m})&\geq p_{L}(y-R_{u}-R_{m})+b\leftrightarrow R_{u}+R_{m}\leq y-\frac{b}{\Delta p} 
    \end{align}
    Los bancos también deberían tener incentivos para hacer el monitoreo;
     \begin{align}
    p_{H}R_{m}-C\geq p_{L}R_{m}\leftrightarrow R_{m}\geq \frac{C}{\Delta p}
    \end{align}
    Ya que el financiamiento bancario es siempre más caro que el financiamiento directo, las firmas prestan lo menos posible a los bancos:
     \begin{align}
    I_{m}&=I_{m}(\beta)
    \end{align}
    Donde se define $I_{m}(\beta)$ como $\frac{p_{H}R_{m}}{\beta}= \frac{p_{H}C}{\beta \Delta p}$
    
\end{frame}

\begin{frame}
    \frametitle{{\normalsize OPCIONES DE FINANCIAMIENTO: FINANCIAMIENTO INDIRECTO} {}}
       Donde $\beta$ denota la tasa esperada de retorno que es demandada por los bancos. La firma podrían obtener el resto de los inversores desinformados.
    \begin{align}
    I_{u}=\frac{p_{H}R_{u}}{1+r}
    \end{align}
    Así que dada la restricción 5 es unida a 4
     \begin{align}
     R_{u}\leq y-\frac{b+C}{\Delta p}
     \end{align}
    Lo cual implica que:
    \begin{align}
    I_{u}\leq\frac{p_{H}}{1+r}(y-\frac{b+C}{\Delta p})
    \end{align}
    De manera que el proyecto puede financiarse si y sólo si:
    \begin{align}
    A+I_{u}+I_{m}\geq I\rightarrow A\geq A^{*}(\beta, r)
    \end{align}
   Donde se define $A^{*}(\beta, r)$ como $1-I_{m}(\beta)-\frac{p_{H}}{1+r}(y-\frac{b+C}{\Delta p})$
   
\end{frame}

\begin{frame}
    \frametitle{{\normalsize OPCIONES DE FINANCIAMIENTO: FINANCIAMIENTO INDIRECTO} {}}
    Finalmente la tasa de retorno $\beta$ está dado por el equilibrio entre la oferta y demanda del capital bancario:
    \begin{align}
    K_{m}=\left[G(\hat{A}(r))-G(A^{*}(\beta, r)) \right] I_{m}(\beta)
    \end{align}
    
    Donde $K_{m}$ denota el capital total de la industria bancaria, $G(\hat{A}(r))-G(A^{*}(\beta, r)$ representa el número (proporción) de las firmas que obtienen créditos y $I_{m}(\beta)$ representa el tamaño del crédito. Noten que el lado derecho de 11 es una función decreciente de $\beta$ y el equilibrio es único. 
\end{frame}

\begin{frame}
    \frametitle{{\normalsize OPCIONES DE FINANCIAMIENTO: FINANCIAMIENTO INDIRECTO} {}}
    \begin{block} {Resultado}
        \begin{itemize}
            \item  En equilibrio sólo las empresas bien capitalizadas ($A\geq \hat{A} $) podrán emitir deuda de manera directa.
            \item  Las firmas con capitalización intermedio ($A^{*}(\beta, r) < A < \hat{A} $) prestarán a los bancos.
            \item Y las firmas subcapitalizadas ($ A \leq A^{*}(\beta, r) $)  no podrán invertir.
        \end{itemize}
      \end{block}
  
\end{frame}

\begin{frame}
\frametitle{{\normalsize OPCIONES DE FINANCIAMIENTO: FINANCIAMIENTO INDIRECTO} {}}
El valor de equilibrio de r (la tasa sin riesgo) y $\beta $ (el retorno bruto de los créditos bancarios), están determinados por dos condiciones:
\begin{itemize}
	\item La ecuación de equilibrio del mercado del capital bancario (11).
	\item La ecuación de equilibrio sobre el mercado financiero, en la que la oferta de ahorro S(r) iguala a la demanda de fondos $D(\beta, r, C)$-
	\begin{align}
	D(\beta, r, C)&=\int_{A(\beta, r)}^{\hat{A}(r)}(I-I_{m}-A)dG(A)+\int_{\hat{A}(r)}^{\hat{A}}(I-A)dG(A)
	\end{align}   
	
\end{itemize}

\end{frame}



\begin{frame}
\frametitle{{\normalsize Extensiones} {}}
\cite{Tirole1997} consideran también un modelo más general; con una variable del nivel de inversión. Ellos estudian los efectos de tres tipos de shocks financieros:
\begin{itemize}
	\item Credit crunch, correspondiente a un decrecimiento de $K_{m}$; el capital de la industria bancaria.
	\item Un colateral squeeze, lo que corresponde a un negativo shock sobre el activo de las firmas.
	\item Ahorro squeeze, lo que corresponde a un desplazamiento contractivo en la función de ahorro.
\end{itemize}
\end{frame}

\begin{frame}
\frametitle{{\normalsize Extensiones} {}}

A partir de lo cual encuentran los siguiente:
\begin{block} {Resultado}
	Dado que $r$ y $\beta$ representan el retorno de equilibrio del mercado financiero y crédito bancario respectivamente, entonces:
	\begin{itemize}
		\item Un credit crunch decrece r y incrementa $\beta$.
		\item Un colateral squeeze decrece r y  $\beta$.
		\item Un ahorro squeeze incrementa r y decrece $\beta$.
	\end{itemize}  
\end{block}
\end{frame}
\section{Riesgo de crédito y dilusión de costo}
\include{9_CLASE}
\section{Competencia perfecta}
\include{10_CLASE}
\section{Monopolio}

\begin{frame}
    \frametitle{{\normalsize MONTI-KLEIN MODEL DE BANCO MONOPOLIO} {}}
    \setcounter{equation}{0}
    \begin{block} {Estructura básica del modelo}
        \begin{description}
            \item[Supuesto 1]  Existe un banco monopolistica, el cual enfrenta una demanda de créditos ($L(r_{L})$) con pendiente negativa y una oferta de depósitos con pendiente positiva ($D(r_{D})$).
            \item[Supuesto 2] La decisión de los bancos es sobre los niveles de L y D.
            \item[Supuesto 3] El banco considera r dado ya sea por que es fijado por el banco central o por el mercado de capital internacional  
            \item[Supuesto 4] La función de beneficios de los bancos ($\pi $) es cóncava.     
        \end{description}
        
    \end{block}	
    
    
    El beneficio del banco puede definirse como:
    \begin{align}
    \pi&=\pi(L, D)=(r_{L}(L)-r)L+(r(1-\alpha)-r_{D}(D))D-C(D,L) 
    \end{align}   
    
    Los beneficios de los bancos es igual al margen de intermediación sobre el crédito y depósitos menos el manejo de costos. 
\end{frame}

\begin{frame}
    \frametitle{{\normalsize MONTI-KLEIN MODEL DE BANCO MONOPOLIO} {}}
    
    Así el comportamientos de los bancos se deducen de las condiciones de primer orden:
    \begin{align}
    \dfrac{\delta \pi}{\delta L}&=r_{L}^{'}(L)L+r_{L}-r-C_{L}^{'}(D,L)=0  \\
    \dfrac{\delta \pi}{\delta D}&=-r_{D}^{'}(D)D+r(1-\alpha)-r_{D}-C_{D}^{'}(D,L)=0  
    \end{align} 
    Si definimos la elasticidad de la demanda de crédito y oferta de depósitos como $\epsilon_{L}=-\frac{r_{L}L^{'}(r_{L})}{L(r_{L})}>0   \; y \; \epsilon_{D}=\frac{r_{D}D^{'}(r_{D})}{D(r_{D})}>0$
    respectivamente. La solución de $r_{L}^{*}\; \; y \; r_{D}^{*}$, es caracterizada por:
    \begin{align}
    \dfrac{r_{L}^{*}-(r+C_{L}^{'})}{r_{L}^{*}}&=\dfrac{1}{\epsilon_{L}(r_{L}^{*})}  \\
    \dfrac{r(1-\alpha)-C_{D}^{'}-r_{D}^{*}}{r_{D}^{*}}&=\dfrac{1}{\epsilon_{D}(r_{D}^{*})}  
    \end{align} 
    
    
    
\end{frame}



\begin{frame}\frametitle{{\normalsize MONTI-KLEIN MODEL DE BANCO MONOPOLIO} {}}
    
     Las dos ecuaciones anteriores son una adaptación simple del sector bancarios su familiaridad es por que el lado izquierdo representa un índice de Lerner (precio menos costos divididos por precio) y el lado derecho la elasticidad inversa.
     Ante un mayor poder de mercado, la elasticidad tiende ha ser pequeño y el índice de Lerner alto. Un modelos competitivo se corresponde a la situación en que la elasticidad tiende ha ser infinita. El resultado nos muestra que los margenes de intermediación tienden a ser alto cuanto más alto sea el poder de mercado.
    {\footnotesize \begin{block} {Resultado}
        \begin{enumerate}
            \item Un banco monopolio es aquel cuyo volumen de depósitos y créditos se ajustan de manera de que el índice de Lerner se iguale a la inversa de la elasticidad. Consecuencia de ello, es que los margenes de intermediación se ven afectado ante la sustitución de un producto bancario por la aparición de un producto en el mercado financiero.
            \item Si el manejo de costos es aditivo, el problema de decisión de los bancos separable. $r_{D}^{*} $ es independiente del mercado de crédito y viceversa
        \end{enumerate}    
   
    \end{block}	 }
    
    
    
\end{frame}




\begin{frame}
    \frametitle{{\normalsize OLIGOPOLIO} {}}
    
    \begin{block} {Estructura básica del modelo}
        \begin{description}
            \item[Supuesto 1]  Existen N bancos indexados por $n=1,2,\cdots N$.
            \item[Supuesto 2] Por simplicidad los bancos tienen una misma función de costos: $C(D,L)=\gamma_{D}D+\gamma_{L}L $.
            \item[Supuesto 3] Definimos el equilibrio de Cournot como la N parejas ($D_{n}^{*}, L_{n}^{*} $) en el que cada una máximiza los beneficios del banco n correspondiente, tomando el volumen de depósitos y créditos de los otros bancos dado.
        \end{description}
       
    \end{block}	
   {\footnotesize  El supuesto 3 nos dice que para un banco n su pareja ($D_{n}^{*}, L_{n}^{*} $), se deduce del problema:
    \begin{align}
    \max_{(D_{n}, L_{n})}  &\left(r_{L}\left(L_{n}+\sum_{m\neq n}L_{m}^{*} \right)-r  \right)L_{n}+ \nonumber \\
    &\left( r(1-\alpha)-r_{D}\left(D_{n}+\sum_{m\neq n}D_{m}^{*} \right) \right)D_{n}-C(D_{n}, L_{n})   \nonumber 
    \end{align}  } 
    
\end{frame}



\begin{frame}
    \frametitle{{\normalsize OLIGOPOLIO} {}}
    
    En ese problema hay un único equilibrio donde: $D_{n}^{*}=\frac{D^{*}}{n}\; \; y\; \; L_{n}^{*}=\frac{L^{*}}{n} $. Las CPO son:
    \begin{align}
    \dfrac{\delta \pi_{n}}{\delta L_{n}}&=r_{L}^{'}(L^{*})\frac{L^{*}}{n}+r_{L}(L^{*})-r-\gamma_{L}=0  \nonumber \\
    \dfrac{\delta \pi_{n}}{\delta D_{n}}&=-r_{D}^{'}(D^{*})\frac{D^{*}}{n}+r(1-\alpha)-r_{D}(D^{*})-\gamma_{D}=0  \nonumber 
    \end{align} 
  Estas CPO pueden escribirse como:
 \begin{align}
 \dfrac{r_{L}^{*}-(r+C_{L}^{'})}{r_{L}^{*}}&=\dfrac{1}{N \epsilon_{L}(r_{L}^{*})}  \\
 \dfrac{r(1-\alpha)-C_{D}^{'}-r_{D}^{*}}{r_{D}^{*}}&=\dfrac{1}{N \epsilon_{D}(r_{D}^{*})}  
 \end{align} 
  Noten que la única diferencia entre el caso de monopolio y Cournot es que la elasticidad es multiplicada por N. Es decir se trata de un modelo de competencia imperfecta con dos casos extremos N=1 y $N=\infty$.      
\end{frame}





\begin{frame}
    \frametitle{{\normalsize OLIGOPOLIO} {}}
    Las ecuaciones 6 y 7, proveen un posible test de competencia imperfecta sobre el sector bancario. Adicionalmente estas ecuaciones permiten ver que la sensibilidad de $r_{L}\; \; y \; \; r_{D}$ ante cambios en r dependerá de N. Lo cual es una aproximación a la intensidad de la competencia.\\
    Asumiendo la elasticidad contante tenemos:
 \begin{align}
 \frac{\delta r_{L}^{*}}{\delta r}&= \dfrac{1}{1-\frac{1}{N \epsilon_{L}}}\nonumber \\
 \frac{\delta r_{D}^{*}}{\delta r} &=\dfrac{1-\alpha}{1+\frac{1}{N \epsilon_{D}}}\nonumber 
 \end{align} 
   Así que cuando la intensidad de la competencia se incrementa $r_{L}^{*}$ es menos sensitivo ante cambios de r.
\end{frame}

\section{Competencia monopolistica}
\include{12_CLASE}
\end{document}