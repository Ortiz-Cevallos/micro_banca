\begin{frame}
	\frametitle{{\normalsize INTRODUCCIÓN} {}}
		\setcounter{equation}{0}
El crédito bancario tiene la cualidad de la \textit{Unicidad} concepto dado por \cite{James1987}, quien notó que el mercado reacciona de manera positiva ante una firma cuando se da cuenta que sus proyectos son financiados en alguna cuota por los bancos.\\

Sin embargo, en los últimos años ha crecido el financiamiento de las firmas de manera directa (proceso de des-intermediación), especialmente entre firmas grandes.\\
En la práctica, la deuda directa es menos costosa que la indirecta, dado que el crédito sólo es buscado por aquellas firmas que no pueden acceder al mercado de deuda directa.\\
Por tanto el objetivo de explicar la coexistencia de estas dos modalidades de financiamiento se ha basado en el problema de \textit{moral hazard}, la idea es evitar aquellas firmas que no cuenten con el suficiente activo para obtener el financiamiento de manera directa.

\end{frame}

\begin{frame}
    \frametitle{{\normalsize UN SIMPLE MODELO DEL MERCADO DE CRÉDITO CON MORAL HAZARD} {}}
    
    \begin{block} {Estructura básica del modelo}
        \begin{description}
            \item[Supuesto 1] Las firmas buscan inversionistas para proyectos cuya inversión sea de costos fijos normalizados en 1.
            \item[Supuesto 2] La tasa libre de riesgo está normalizada a 0 ($r=0$). 
            \item[Supuesto 3] Las firmas tienen que escoger entre los siguientes proyectos:
            \begin{enumerate}
               \item \textit{Tecnología buena.} la cual produce un suceso G con una probabilidad de $\pi_{G} $ y cero en el otro caso.
               \item \textit{Tecnología mala.} la cual produce un suceso B con una probabilidad de $\pi_{B} $ y cero en el otro caso.
            \end{enumerate}
            \item[Supuesto 4] Sólo los buenos proyectos tienen un valor presente neto positivo. Es de notar que: $\pi_{G}G>1>\pi_{B}B $, pero $B>G\rightarrow \pi_{G}>\pi_{B} $; esto último significa que a mayor rentabilidad menor probabilidad de ocurrencia. 
           
             
            \end{description}
        
    \end{block}	
    
\end{frame}


\begin{frame}
    \frametitle{{\normalsize UN SIMPLE MODELO DEL MERCADO DE CRÉDITO CON MORAL HAZARD} {}}
    
    \begin{block} {Estructura básica del modelo}
        \begin{description}
             \item[Supuesto 5] Asuma que el suceso de la inversión es verificable por externos; pero no la elección de tecnología de las firmas ni su retorno. Por tanto la firma puede hacer una promesa de pago de un monto fijo R sólo en caso se da el suceso de la inversión.
            \item[Supuesto 6] Las firmas no tienen otra fuente de recursos, así que no paga la promesa en caso la inversión falla.
            
        \end{description}
          
    \end{block}	

 
\end{frame}

\begin{frame}
\frametitle{{\normalsize UN SIMPLE MODELO DEL MERCADO DE CRÉDITO CON MORAL HAZARD} {}}

Es de notar que el valor R del endeudamiento de las firmas determina la elección de la tecnología. Adicionalmente en ausencia de monitoreo, la firma escoge la tecnología buena si y sólo si esta produce un beneficio esperado alto:
\begin{align}
\pi_{G}(G-R)&>\pi_{B}(B-R) \\
ó \nonumber \\
R<R_{C}&=\frac{\pi_{G}G-\pi_{B}B}{\pi_{G}-\pi_{B}}    
\end{align}
Donde $R_{C}$ denota el valor crítico de la deuda nominal con la cual si R es mayor que $R_{C}$ la firma escoge la mala tecnología. 
\end{frame}

\begin{frame}
    \frametitle{{\normalsize UN SIMPLE MODELO DEL MERCADO DE CRÉDITO CON MORAL HAZARD} {}}
 Desde el punto de vista de la firma que toma el crédito, la probabilidad de pagar dicho compromiso depende de R:
 
 \[\pi(R)=\left\{ \begin{array}{rcl}
 \pi(G) & \mbox{si} & R\leq R_{C}\\
 & & \\
 \pi(B) & \mbox{si} & R > R_{C}\\
 \end{array}
 \right. \] 

En ausencia de monitoreo, el equilibrio competitivo del mercado de crédito se obtiene un R tal que:
  \begin{align}
  \pi(R)R=1
  \end{align}    
Ya que $\pi_{B}R<1$ para todo $R\leq B$. El equilibrio es solo posible cuando la tecnología buena es implementada, implicando que  $R<R_{C}$ y $\pi_{G}R_{C}\geq 1$ esto sólo se satisface si el moral hazard no es muy importante.\\
Si  $\pi_{G}R_{C}<1$, el equilibrio implica una ausencia de comercio y por tanto el mercado de crédito colapsa, dado que los proyectos buenos no son financiado y los malos tienen un valor presente neto esperado negativo.

\end{frame}

\begin{frame}
\frametitle{{\normalsize UN SIMPLE MODELO DEL MERCADO DE CRÉDITO CON MORAL HAZARD} {}}
Introduciendo una tecnología de monitoreo:\\
Con un costo C un banco puede prevenir a las firmas a utilizar la mala tecnología- Suponiendo competencia perfecta entre bancos, el valor nominal de los créditos bancarios en equilibrio (denotado por $R_{m}$ donde m denota al monitor) está dada por la condición de quiebre:
\begin{align}
\pi_{G}R_{m}&=1+C
\end{align}   
\end{frame}

\begin{frame}
\frametitle{{\normalsize UN SIMPLE MODELO DEL MERCADO DE CRÉDITO CON MORAL HAZARD} {}}
   
Para que el crédito bancario se encuentre en equilibrio, dos condiciones son necesarias:
\begin{enumerate}
	\item El reembolso nominal del crédito bancario en equilibrio tiene que ser menor que el retorno G que reciben las firmas. Ello es equivalente a:
	\begin{align}
	\pi_{G}G-1&>C
	\end{align} 
	Es decir que el costo de monitoreo tiene que ser menor al valor presente neto del proyecto bueno 
	\item El financiamiento directo, el cual es menos costoso, tiene que ser imposible
	\begin{align}
	\pi_{G}R_{C}&<1
	\end{align} 
\end{enumerate}

\end{frame}


\begin{frame}
    \frametitle{{\normalsize UN SIMPLE MODELO DEL MERCADO DE CRÉDITO CON MORAL HAZARD} {}}
   Así el crédito bancario aparece en equilibrio para valores intermedios en probabilidad
    \begin{align}
    \pi_{G}(\pi_{G} \in \left[\frac{1+C}{G},\; \frac{1}{R_{C}}\right] )\nonumber
    \end{align} 
    el cual es un intervalo que no es vacío.
    
\end{frame}

\begin{frame}
\frametitle{{\normalsize UN SIMPLE MODELO DEL MERCADO DE CRÉDITO CON MORAL HAZARD} {}}
Por lo que se debe de establecer lo siguiente:
\begin{block} {Resultado} 
	Suponga que el costo de monitores es los suficientemente pequeño que $\frac{1}{R_{C}}>\frac{1+C}{G} $. Entonces hay tres posibles regímenes de mercado de crédito de equilibrio:
	\begin{enumerate}
		\item Si $\pi_{G}>\frac{1}{R_{C}}$ suceso altamente probable, las firmas piden prestado a los bancos a una tasa: $R_{1}=\frac{1}{\pi_{G}}$.
		\item  Si $\pi_{G}\in \left[\frac{1+C}{G},\; \frac{1}{R_{C}}\right] $ suceso medianamente probable, las firmas piden prestado a los bancos a una tasa: $R_{2}=\frac{1+C}{\pi_{G}}$. 
		\item  Si $\pi_{G} < \frac{1+C}{G} $ suceso poco probable, el mercado de crédito colapsa.
	\end{enumerate}    
\end{block}    

\end{frame}

