\documentclass[10pt, xcolor=table, x11names]{beamer}
\usepackage[spanish]{babel} %CORTE DE PALABRAS RESPETANDO EL IDIOMA ESPAÑOL.
\usepackage[Utf8]{inputenc} %acentos desde el teclado
\usepackage	{textpos}
\usepackage{tikz}
\usetikzlibrary{arrows,positioning} 
\usefonttheme{professionalfonts} % fuentes de LaTeX\epsilon
\usetheme{Boadilla}      % or try Darmstadt, Madrid, Warsaw, ...
\usecolortheme[RGB={130,130,190}]{structure} % or try albatross, beaver, crane, ...
\useinnertheme{rounded}
%\useoutertheme{shadow}
\setbeamertemplate{blocks}[rounded][shadow=true]
\setbeamertemplate{navigation symbols}{}
\setbeamercovered{transparent} % Velos
\setbeamertemplate{caption}[numbered]
\usepackage[spanish, authoryear, roud, datebegin]{flexbib} %CITAS BIBLIOGRÁFICAS
\newtheorem{Teorema}{Teorema}
\usepackage{ragged2e}
\justifying
\usepackage{booktabs}
\usepackage{multirow}
\usepackage[x11names,table]{xcolor}
\usepackage[pdftex]{graphicx}
\usepackage{epstopdf} % Convertir .eps a .pdf (si fuera necesario)
\DeclareGraphicsExtensions{.pdf,.png,.jpg, .eps} % busca en este orden!
\title[]{INTRODUCCIÓN}
\author[Luis Ortiz]{Luis Ortiz Cevallos}
\institute[SECMCA]{\bf SECMCA}
\date[\today]{\footnotesize \today}
\usepackage[pdftex]{hyperref}
\hypersetup{colorlinks,%
	citecolor=blue,%
	filecolor=blue,%
	linkcolor=blue,%
	urlcolor=blue,%
	pdftex}

\begin{document}


\begin{frame}
\titlepage
\end{frame}




\begin{frame}
	\frametitle{{\normalsize INTRODUCCIÓN} {}}
	¿Qué es un banco y qué es lo que hace?\\
    Un banco es una institución cuya actividad {\bf\emph{corriente}} es conceder crédito y recibir depósitos del {\bf\emph{público}}. \\
    Los bancos financian una importante fracción de sus créditos a través de los depósitos del público, ahí la principal explicación de su fragilidad y la justificación de su regulación. Por ello algunos economistas predicen que los bancos serán sustituidos por los fondos mutuos o narrow banking, quienes invierten los depósitos en valores negociados o por otras instituciones financieras quienes conceden crédito a través de la emisión de deuda o acciones.\\
    El término {\bf\emph{público}}, enfatiza que los bancos provee un único servicio al público: Liquidación y medio de pago. Es de notar que el público a diferencia de los inversores institucionales, no tiene los medios para evaluar la solidez de una institución y la calidad de sus activos, por lo que confían en los bancos para proveerse de esos bienes públicos. 
     
    
    	
\end{frame}

\begin{frame}
    \frametitle{{\normalsize FUNCIONES DE UN BANCO} {}}
   Los bancos desarrollan cuatro funciones:
   \begin{itemize}
       \item Ofrecen liquidez y servicios de pagos.
       \item Transforman activos
       \item Administran riesgos
       \item Procesan información y monitorean a los deudores
   \end{itemize}    
\end{frame}

\begin{frame}
    \frametitle{{\normalsize Liquidez y servicios de pagos} {}}
    Dada la existencia de costos de transacción el dinero es el medio de cambio. Hay dos tipos de dinero:
    \begin{itemize}
        \item Dinero mercancía 
        \item Dinero fiduciario
    \end{itemize}
    Esta función de los bancos se puede entender de forma más precisa en dos actividades:
    \begin{itemize}
        \item Cambio de moneda
        \item Servicios de pagos
    \end{itemize}
\end{frame}

\begin{frame}
    \frametitle{{\normalsize Transformación de activos} {}}
    La transformación de activos puede ser a través de 3 perspectivas:
    \begin{itemize}
        \item Conveniencia de denominación (unidad de tamaño)
        \item Transformación de calidad (motivado por: Indivisibilidad de inversión, cuando pequeños depositarios no pueden diversificar su portafolio e información asimétrica a favor de los bancos)
        \item Transformación de madurez (esto implica un riesgo que es mitigado por el crédito interbancario y derivado e instrumentos financieros )
     \end{itemize}
\end{frame}
	
    \begin{frame}
        \frametitle{{\normalsize Administración de riesgos} {}}
     Los riesgos usuales que enfrentan los bancos corresponden a una línea de sus balances. Estos son:
     \begin{itemize}
         \item Riesgo de crédito
         \item Riesgo de tasa de interés
         \item Riesgo de liquidez
     \end{itemize}
     Adicionalmente existe otro riesgo que no se identifica en la hoja de balance de los bancos pero que esta surgiendo en las últimas décadas:
     \begin{itemize}
         \item Riesgo por operaciones fuera de balances
     \end{itemize}
\end{frame}
    
\begin{frame}
    \frametitle{{\normalsize Bancos en el modelo Arrow-Debreu: Esquema de las decisiones económicas de los agentes} {}}
    
    \begin{tikzpicture}
    \node[right] (MF) at (0.0,10) {{\tiny MERCADO FINANCIERO}};
    \node[right] (MF1) at (0.2,9.5){{\tiny $B_{f}+B_{b}=B_{h}$}};
    
    \node[right] (F) at (-3.3,7.5) {{\tiny $Firmas$}};
    \node[left] (F1) at (-3.0,7.3) {{\tiny $Activos$}};
    \node[right] (F2) at (-2.5,7.3){{\tiny $Pasivos$}};
    \node (F11) at (-3.5,7.0)  {{\tiny $Incersión I$}};
    \node (F21) at (-1.7,7.0) {{\tiny Títulos $B_{f}$}};
    \node (F22) at (-1.4,6.75) {{\tiny Créditos $ L^{-}$}};
    
    \node[right] (B) at (0.5,5.0) {{\tiny $Bancos$}};
    \node[left] (B1) at (0.75,4.75) {{\tiny $Activos$}};
    \node[right] (B2) at (1.0,4.75){{\tiny $Pasivos$}};
    \node (B11) at (-0.25,4.5)  {{\tiny  Créditos $ L^{+}$}};
    \node (B21) at (2.25,4.5) {{\tiny Títulos $B_{b}$}};
    \node (B22) at (3.0,4.25) {{\tiny Dpósitos $ D^{-}$}};
    
\node[right] (H) at (3.3,7.5) {{\tiny $Hogares$}};
    \node[left] (H1) at (3.75,7.3) {{\tiny $Activos$}};
\node[right] (H2) at (4.25,7.3){{\tiny $Pasivos$}};
\node[left] (H11) at (3.5,7.0)  {{\tiny Títulos $B_h$}};
\node[left] (H12) at (3.5,6.75)  {{\tiny Depósitos $D^{+}$}};
\node[right] (H21) at (4.5,7.0) {{\tiny Ahorros $S$}};

    
    
    \path[->] (MF1) edge (F21)
    (MF1) edge (B21)
    (B11) edge (F22)
    (H12) edge (B22)
    (H11) edge (MF1);
    \end{tikzpicture}
\end{frame}


\begin{frame}
    \frametitle{{\normalsize Bancos en el modelo Arrow-Debreu} {}}
   \begin{block} {Objetivo}
       Conocer la utilidad de los bancos en el modelo Arrow-Debreu. 	
   \end{block}	
   \begin{block} {Estructura: Hogares}
       \begin{description}
           \item[Supuesto 1] Los hogares viven en esa economía por 2 períodos y estan dotados de una riqueza inicial: $W_{1}$.
           \item[Supuesto 2] Los hogares seleccionan el perfil temporal de su consumo: $(C_{1}, C_{2})$
           \item[Supuesto 3] Los hogares son los dueños de los bancos y empresas, reciben el periodo 2 los beneficios que las empresas y bancos obtuvieron: $\Pi_{f}$ y $\Pi_{b}$ respectivamente. 
           \item[Supuesto 4] Los hogares deciden como mantener sus ahorro entre tres opciones: Ahorro en el banco en forma  de depósitos $D^{+}$ cuyo rendimiento es $r_{b}$, títulos en bonos de empresas $B_{f} $, cuyo rendimiento es $r_{f} $ y títulos en bancos $B_{h}$ cuyo rendimiento es $r_{b}$. 	
          \end{description}
     
   \end{block}	
\end{frame}



\begin{frame}
    \frametitle{{\normalsize Bancos en el modelo Arrow-Debreu} {}}
    
    \begin{block} {Estructura: Hogares} 
     Bajo esa estructura el problema de los hogares se resumen en:
    \begin{align}
    \max U(C_{1},C_{2})\nonumber \\
    s.a:\nonumber\\
    C_{1}+B_{h}+D_{h}&=W_{1}\nonumber\\      pC_{2}&=\Pi_{f}+\Pi_{b}+(1+r)B_{h}+(1+r_{b})D_{h}\nonumber
    \end{align}
    \end{block}
      \begin{block} {Implicaciones}
        La solución de la cartera de ahorro de los hogares es interior si y sólo sí se cumple que:
        \begin{align}
        r&=r_{b}
        \end{align} 	
       
    \end{block}	

\end{frame}



\begin{frame}
    \frametitle{{\normalsize Bancos en el modelo Arrow-Debreu} {}}
     \begin{block} {Estructura: Empresas}
        \begin{description}
            \item[Supuesto 1] Las firmas escogen su nivel de inversión: I y deciden como financiarla: Solicitando Crédito a los bancos: $L_{f}$, o emitiendo títulos de deudo con los hogares $B_{f}$
            \end{description}
        Bajo esa estructura el problema de las empresas es el de máximizar sus beneficios:
        \begin{align}
        \max \Pi_{f}\nonumber \\
        s.a:\nonumber\\
        \Pi_{f}&= pF(I)-(1+r)B_{f}-(1+r_{L})L_{f}\nonumber\\      
        I&=B_{f}+L_{f}\nonumber
       \end{align}
    \end{block}	
    \begin{block} {Implicaciones}
        Si la emisión de títulos y el Crédito son sustitutos perfectos se obtiene una solución interior por que se cumple que:
        \begin{align}
        r&=r_{L}
        \end{align} 
     \end{block}
   
\end{frame}



\begin{frame}
    \frametitle{{\normalsize Bancos en el modelo Arrow-Debreu} {}}
    \begin{block} {Estructura: Bancos}
        \begin{description}
            \item[Supuesto 1] Los bancos escogen su oferta de créditos: $L_{b}$, su demanda de depósitos: $D_{b}$ y su emisión de títulos: $B_{b} $.
        \end{description}
        Bajo esa estructura el problema de los bancos es el de máximizar sus beneficios:
        \begin{align}
        \max \Pi_{b}\nonumber \\
        s.a:\nonumber\\
        \Pi_{b}&= r_{L}L_{b}-rB_{b}-r_{d}D_{b}\nonumber\\      
        L&=B_{b}+D_{b}\nonumber
        \end{align}
    \end{block}	
 \end{frame}


\begin{frame}
    \frametitle{{\normalsize Bancos en el modelo Arrow-Debreu} {}}
    \begin{block} {Equilibrio General}
    El equilibrio general esta caracterizado por los vectores: $ (r, r_{L}, r_{D}) $ y tres vectores adicionales de la demanda y oferta de los hogares ($(C_{1}, C_{2}, B_{h}, D_{h}) $), empresas ($(I, B_{f}, L_{f}) $) y bancos ($(L_{b}, B_{b}, D_{b}) $).
    Teniendo en cuenta:
    \begin{itemize}
        \item Cada agente se comporta optimamente.
        \item Cada mercado se clarea:
        \begin{itemize}
            \item $I=S$ (mercados de bienes)
            \item $D_{h}=D_{b} $ (mercado de depósitos)
            \item $L_{f}=L_{b} $ (mercado de créditos)
            \item $B_{H}=B_{f}+B_{b}$ (mercado de bonos)
        \end{itemize}
    \end{itemize}
   Dada las ecuaciones 1 y 2 está claro que una de las condiciones de equilibrios es que:
   
   \begin{align}
   r&=r_{L}&=r_{b}
   \end{align} 
    \end{block}	
\end{frame}



\begin{frame}
    \frametitle{{\normalsize Bancos en el modelo Arrow-Debreu} {}}
    \begin{block} {Implicaciones}
        \begin{enumerate}
            \item La condición de equilibrio provoca que los beneficios de los bancos sean cero 
            \item Tanto los hogares como las firmas no enfrentan restricciones a un mercado financiero perfecto
           \item El tamaña de los balances bancarios no tienen ningun efecto en otros agentes económicos
           \item El modelo de equilibrio general con mercado financieros completos (el modelo Arrow-Debreu) no pueden ser usado para el estudio del sector bancario (los bancos son redundantes). Hay dos vías para elaborar un modelo util para el análisis de los bancos. Estos son
           \begin{itemize}
               \item El paradigma del mercado incompleto
               \item El recurso de la Organización industrial de los bancos.
           \end{itemize}
        \end{enumerate}
     
    \end{block}	
\end{frame}

\end{document}