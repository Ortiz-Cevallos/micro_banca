\documentclass[10pt, xcolor=table, x11names]{beamer}
\usepackage[spanish]{babel} %CORTE DE PALABRAS RESPETANDO EL IDIOMA ESPAÑOL.
\usepackage[Utf8]{inputenc} %acentos desde el teclado
\usepackage	{textpos}
\usepackage{tikz}
\usetikzlibrary{arrows,positioning} 
\usefonttheme{professionalfonts} % fuentes de LaTeX\epsilon
\usetheme{Boadilla}      % or try Darmstadt, Madrid, Warsaw, ...
\usecolortheme[RGB={130,130,190}]{structure} % or try albatross, beaver, crane, ...
\useinnertheme{rounded}
%\useoutertheme{shadow}
\setbeamertemplate{blocks}[rounded][shadow=true]
\setbeamertemplate{navigation symbols}{}
\setbeamercovered{transparent} % Velos
\setbeamertemplate{caption}[numbered]
\usepackage[spanish, authoryear, roud, datebegin]{flexbib} %CITAS BIBLIOGRÁFICAS
\newtheorem{Teorema}{Teorema}
\usepackage{ragged2e}
\justifying
\usepackage{booktabs}
\usepackage{multirow}
\usepackage[x11names,table]{xcolor}
\usepackage[pdftex]{graphicx}
\usepackage{epstopdf} % Convertir .eps a .pdf (si fuera necesario)
\DeclareGraphicsExtensions{.pdf,.png,.jpg, .eps} % busca en este orden!
\title[]{ENFOQUE DE ORGANIZACIÓN INDUSTRIA (OI): INTRODUCCIÓN}
\author[Luis Ortiz]{Luis Ortiz Cevallos}
\institute[SECMCA]{\bf SECMCA}
\date[\today]{\footnotesize \today}
\usepackage[pdftex]{hyperref}
\hypersetup{colorlinks,%
	citecolor=blue,%
	filecolor=blue,%
	linkcolor=blue,%
	urlcolor=blue,%
	pdftex}

\begin{document}


\begin{frame}
\titlepage
\end{frame}




\begin{frame}
	\frametitle{{\normalsize INTRODUCCIÓN} {}}
OBJETIVOS:
\begin{itemize}
    \item Estudiar las implicaciones de la teoría estándar de IO sobre el comportamiento de los bancos, con el fin de clarificar las nociones de precio competitivo, poder de mercado y competencia monopolistica.
    \item Estudiar características específicas de los bancos sobre precios y cantidades de equilibrio, lo que significa considerar un amplio set de variables y estrategias disponibles por los bancos; aquí se asoman las siguientes pregutas:
    \begin{enumerate}
        \item ¿Las tasas de interés de los bancos es pegajosa, es decir varía menos que la tasa de interés de mercado \cite{bibid}.
        \item A pesar del desarrollo del internet, ¿la distancia entre un banco y una firma aún importa \cite{bibid}-
        \item ¿Cuál es La posibilidad de un racionamiento del crédito \cite{bibid}.?
        \item La existencia de ``un curso de ganadores'' que hace que el crédito bancario hacia nuevo emprendedores sea menos beneficioso para entrantes que para los encubadores \cite{bibid}.
          
    \end{enumerate}
\end{itemize}
\end{frame}



\begin{frame}
    \frametitle{{\normalsize INTRODUCCIÓN} {}}
      Los supuestos generales de los modelos del enfoque IO son los siguientes:
      \begin{itemize}
          \item Los bancos son definidos como intermediarios financieros 
          \item La tecnología bancaría se encuentra dada
      \end{itemize}
\end{frame}

\begin{frame}[allowframebreaks]
\frametitle{{\large 
		Bibliografía}}
\renewcommand{\refname}{Referencias}
\bibliography{Biblioteca}
\bibliographystyle{flexbib}
\end{frame}

\end{document}